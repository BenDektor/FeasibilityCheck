% arara: pdflatex: { shell: yes } until !found('log', '\\(?(R|r)e\\)?run (to get|LaTeX)')
\documentclass[fontsize=12pt,
parskip=half,	% Abstände statt Einrückungen bei Absätzen
department=FakM,  % Farbanpassungen
twoside, % Spart Papier und erhöht die Lesbarkeit
DIV=15,BCOR=10mm, % Seitenlayout wie bei Koma-Script
]{OTHRreprt}
\usepackage[utf8x]{inputenc}
\usepackage[english,ngerman]{babel} % German documents with some fragments of English
%%\usepackage[ngerman,english]{babel} % Use instead for English documents
%%\usepackage[svgnames, table]{xcolor} % Will work only using recent LaTeX cores
\usepackage{acronym} % Abkürzungsverzeichnis
\usepackage[bookmarks, raiselinks, pageanchor, hyperindex, colorlinks, hidelinks]{hyperref}

%\usepackage{amsmath}				% Pakete fuer den Mathematikmodus
%\usepackage{amssymb}
\usepackage{pifont}				% zusaetzliche Symbole


\usepackage[format=hang,			% Einstellung fuer Bildunterschriften
font={footnotesize},
labelfont={bf},
margin=1cm,
aboveskip=5pt,
position=bottom]{caption}

\usepackage{booktabs} % Hübschere Tabellen
%\usepackage{tikz}								% Erstellen von Grafiken
%\usetikzlibrary{positioning,arrows,plotmarks} 	% TikZ-Bibliotheken
\usepackage[autostyle=true,german=quotes]{csquotes}	% Zur Nutzung von deutschen Anführungszeichen, innerhalb des Textes mit dem Befehl \enquote vorgehen
\usepackage[bottom]{footmisc}
\usepackage[gen]{eurosym}				% Eurozeichen einfügen
%\usepackage{chngpage}					
%\usepackage{lscape}						% Nützlich, falls querformatierte 	Seiten gewünscht sind
%\usepackage{pdflscape}					% Zum exportieren der Landscapes in PDF-Dateien
\usepackage{lipsum}						% Dummy-Text generieren

\usepackage[headsepline]{scrlayer-scrpage}
\pagestyle{scrheadings}
\automark[chapter]{chapter}
\automark*[section]{}

\documenttype{Bachelorarbeit}
\title{Mustertitel der Musterarbeit}
\author{Max Musterman}
\studentid{12345678}
\department{Musterfakultät}
\studyprogramme{Bachelor Musterstudiengang}
\startingdate{1.\,November 2088}
\closingdate{11.\,Dezember 2089}
\firstadvisor{Prof. Dr. Paula Strikt}
\secondadvisor{Prof. Dr. Petra Hart}
\externaladvisor{Dr. Klara Endlos, ACME Inc.}

%\externallogo[height=1.5cm]{firmenlogo}

% Hiermit trägt pdflatex die PDF-Metadaten des erzeugten Dokuments ein:
\hypersetup{pdftitle={\csname @title\endcsname{}},%
	pdfauthor={\csname @author\endcsname{}},%
	%pdfsubject={Optionaler Untertitel / englischer Titel},%
	%pdfkeywords={Optionale Schlüsselwörter}
	}


\begin{document}
	\maketitle
	\cleardoublepage
	\pagenumbering{roman}
	\begin{abstract}
	\section*{Kurzzusammenfassung}
	\begin{quote}
	\lipsum[1]
	\end{quote}
	\end{abstract}
	\cleardoublepage
	\tableofcontents
	\cleardoublepage		
	\pagenumbering{arabic}
		
	%\include{Kapitel1}	% Die Kapitel als seperate .tex Datei im Ordner abspeichern. Dort dann Befehle wie \chapter{} und \section{} sowie \subsection{} verwenden (Keine neue "documentclass!")
	%\include{Kapitel2}
	
	%%% Die folgenden Zeilen dienen nur zur Veranschaulichung des Textlayouts, sie sollten später gelöscht werden!
	\chapter{Mustereinleitung}
	\lipsum[1]

	Manchmal ist es nicht ganz zu vermeiden, Abkürzungen wie \ac{I2C} zu verwenden. 
	Diese sollten dann in einem Abkürzungsverzeichnis (siehe in diesem Dokument auf Seite~\pageref{Sec:Abkuerzungen}) zu finden sein, damit sie bei späterer Verwendung wie hier \ac{I2C} auch nachgeschlagen werden können.

	\lipsum[2-4]
	\section{Musterüberschrift}
	\lipsum[5-7]
	\subsection{Musterunterpunkt}
	\lipsum[8-9]
	\chapter{Musterhauptteil}
	\lipsum[1]
	
	
	% Literaturverzeichnis in das Inhaltsverzeichnis einfügen
	\addcontentsline{toc}{chapter}{Literaturverzeichnis}
	
	% Style für die Bibliothek festlegen
  	\bibliographystyle{plain}
  	
  	% Einfügen des Literaturverzeichnisses in das Dokument
	%\bibliography{Musterpfad/Zum/Literaturverzeichnis}
	
	\newpage
		
	\thispagestyle{empty}	
	\addcontentsline{toc}{chapter}{Abkürzungsverzeichnis}\label{Sec:Abkuerzungen}
	\chapter*{Abkürzungsverzeichnis}
	\begin{acronym}[OTH R]
	 \acro{OTH R}{Ostbayerische Technische Hochschule Regensburg}
	 \acro{OTH}{Ostbayerische Technische Hochschule}
	 \acro{GCC}{Der Compiler namens gcc}
	 \acro{I2C}[I²C]{Inter-Integrated Circuit}
	\end{acronym}
	
	% Anhang
	\addcontentsline{toc}{chapter}{Abbildungsverzeichnis}
	\listoffigures
	
	\addcontentsline{toc}{chapter}{Tabellenverzeichnis}
	\listoftables
	
	%\addcontentsline{toc}{chapter}{Listingsverzeichnis} % für Quellcode
	%\lstlistoflistings 
	
	%\addcontentsline{toc}{chapter}{Digitaler Anhang}	% für digitalen Anhang, falls nötig
	%\include{anhang}
	\cleardoublepage
	\thispagestyle{empty}
	\makedeclaration

	\cleardoublepage

\end{document}
