% arara: pdflatex: { shell: yes } until !found('log', '\\(?(R|r)e\\)?run (to get|LaTeX)')
\documentclass[fontsize=12pt,
parskip=half,	% Abstände statt Einrückungen bei Absätzen
department=FakM,  % Farbanpassungen
twoside, % Spart Papier und erhöht die Lesbarkeit
DIV=15,BCOR=10mm, % Seitenlayout wie bei Koma-Script
]{OTHRreprt}
\usepackage[utf8x]{inputenc}
\usepackage[english,ngerman]{babel} % German documents with some fragments of English
%%\usepackage[ngerman,english]{babel} % Use instead for English documents
%%\usepackage[svgnames, table]{xcolor} % Will work only using recent LaTeX cores
\usepackage{acronym} % Abkürzungsverzeichnis
\usepackage[bookmarks, raiselinks, pageanchor, hyperindex, colorlinks, hidelinks]{hyperref}

%\usepackage{amsmath}				% Pakete fuer den Mathematikmodus
%\usepackage{amssymb}
\usepackage{pifont}				% zusaetzliche Symbole


\usepackage[format=hang,			% Einstellung fuer Bildunterschriften
font={footnotesize},
labelfont={bf},
margin=1cm,
aboveskip=5pt,
position=bottom]{caption}

\usepackage{booktabs} % Hübschere Tabellen
%\usepackage{tikz}								% Erstellen von Grafiken
%\usetikzlibrary{positioning,arrows,plotmarks} 	% TikZ-Bibliotheken
\usepackage[autostyle=true,german=quotes]{csquotes}	% Zur Nutzung von deutschen Anführungszeichen, innerhalb des Textes mit dem Befehl \enquote vorgehen
\usepackage[bottom]{footmisc}
\usepackage[gen]{eurosym}				% Eurozeichen einfügen
%\usepackage{chngpage}					
%\usepackage{lscape}						% Nützlich, falls querformatierte 	Seiten gewünscht sind
%\usepackage{pdflscape}					% Zum exportieren der Landscapes in PDF-Dateien

\usepackage[headsepline]{scrlayer-scrpage}
\pagestyle{scrheadings}
\automark[chapter]{chapter}
\automark*[section]{}

\documenttype{Bachelorarbeit}
\title{Entwicklung eines automatisierten technischen Feasibility Checks für die Durchführung von Qualifikationen neuer Halbleiter-Produkte}
\author{Benedikt Schedlbauer}
\studentid{3322954}
\department{Mathematik und Informatik}
\studyprogramme{Bachelor Technische Informatik}
\startingdate{1.\,Oktober 2024}
\closingdate{28.\,Februar 2025}
\firstadvisor{Prof. Dr. Alexander Metzner}
\secondadvisor{Prof. Dr. Daniel Münch}
\externaladvisor{Florian Saller, Infineon Technologies AG}

%\externallogo[height=1.5cm]{firmenlogo}

% Hiermit trägt pdflatex die PDF-Metadaten des erzeugten Dokuments ein:
\hypersetup{pdftitle={\csname @title\endcsname{}},%
	pdfauthor={\csname @author\endcsname{}},%
	%pdfsubject={Optionaler Untertitel / englischer Titel},%
	%pdfkeywords={Optionale Schlüsselwörter}
	}


\begin{document}
	\maketitle
	\cleardoublepage
	\pagenumbering{roman}
	\begin{abstract}
	\section*{Kurzzusammenfassung}
	\begin{quote}
	\end{quote}
	\end{abstract}
	\cleardoublepage
	\tableofcontents
	\cleardoublepage		
	\pagenumbering{arabic}
		
	%\include{Kapitel1}	% Die Kapitel als seperate .tex Datei im Ordner abspeichern. Dort dann Befehle wie \chapter{} und \section{} sowie \subsection{} verwenden (Keine neue "documentclass!")
	%\include{Kapitel2}
	
	%%% Die folgenden Zeilen dienen nur zur Veranschaulichung des Textlayouts, sie sollten später gelöscht werden!
	\chapter{Einleitung}

\section{Motivation}
Die Idee der vorliegenden Arbeit entstand mit der Abteilung CSC FI OES LMS der Infineon Technologies AG. 
In der Abteilung arbeite ich schon seit circa zwei Jahren als Werkstudent, an Themen wie Web und Mobile-App-Entwicklung unter meinem Betreuer Florian Saller und meinem Zweibetreuer Fabian Vilsmeier.
Zusammen mit diesen, und meinem Auftraggeber Thomas Gombocz, der bei Infineon in München sitzt, habe ich mein Thema des automatisierten technischen Feasibility Checks für die Durchführung von Qualifikationen neuer Halbleiter-Produkte ausgearbeitet.


\section{Zielsetzung der Arbeit}
In dieser Bachelorarbeit wird ein sogenannter technischer Feasibility Check, zu deutsch Machbarkeitsstudie, geplant und umgesetzt. Dieser ist Bestandteil einer größeren internen Software-Applikation namens \ac{REALIS}(diese wird genauer beschrieben in Kapitel~\ref{Sec:REALIS}), welches
im Zuge einer Migration \todo{was für eine Migration?} von einer Windows-Applikation zu einer Web-Applikation zusätzlich verbessert und modernisiert werden soll. Dabei ist der Feasibility Check ein neues Feature, dass zuvor manuell von Anwendern durchgeführt wurde, und nun automatisiert werden soll.

Der Feasibility Check ist eine Backend-Logik, geschrieben in der Programmiersprache C\#, die durch einen einfachen HTTP-Aufruf auf dem Frontend, einer Website, gestartet wird. Dabei greift der Algorithmus auf Daten in der Datenbank zu und liefert ein Ergebnis zurück.

Neben der Backend-Entwicklung werden in dieser Arbeit auch ein erweitertes Datenbankmodell konzipiert und eine erste Schnittstelle mit Angular auf der Website programmiert. \todo{was noch alles?}

\section{Software-Entwicklungsprozess}
Software-Entwicklungsprozesse werden durch sogenannte Vorgehensmodelle beschrieben. Das Ziel solcher Modelle ist es, eine Hilfestellung bei der Organisation von Software-Entwicklungsprojekten zu geben und die Menge aller dabei anfallenden Aktivitäten in klare und verbindliche Arbeitsschritte aufzuteilen. 

Für dieses Projekt wird das Prototyping-Modell eingesetzt. In diesem Ansatz werden schrittweise Prototypen basierend auf den aktuellen Anforderungen entwickelt. Das daraufhin eingeholte Feedback von Auftraggebern oder Endanwendern ermöglicht eine kontinuierliche Verfeinerung der Anforderungen und eine schrittweise Verbesserung des Prototyps. Die Stadien dieses Modells werden in Abbildung~\ref{fig:Prototyping-Modell} nochmal verdeutlicht \cite{senarath2021waterfall}.

\begin{figure}[h!]
    \centering
    \includegraphics[]{bilder/Prototyping_Stages.jpg}
    \caption{Prototyping-Modell Stadien}
    \label{fig:Prototyping-Modell}
\end{figure}

Das Prototyping-Modell wurde eingesetzt, da die Anforderungen des technischen Feasibility Checks zu Beginn noch nicht eindeutig definiert waren und erst im Laufe der Entwicklung bestimmte Aspekte geklärt werden konnten. Zudem förderte dieser Ansatz den regelmäßigen Austausch zwischen mir als Entwickler und den Auftraggebern.

\section{Aufbau der Arbeit}
Die Bachelorarbeit ist so strukturiert, dass nach diesem einleitenden Kapitel in Abschnitt \ref{Chap:TheoretischeGrundlagen} zuallererst, die wichtigsten theoretischen Grundlagen geklärt werden. 
Hierbei wird kurz Infineon und die Halbleitertechnologie erläutert, woraufhin auf die Software-Applikation \ac{REALIS} eingegangen wird und anschließend kurz erklärt wird, was der englische Titel "Feasibility Check" überhaupt meint. Auch wird noch auf die in der Bachelorarbeit verwendeten Programmiersprachen bzw. Software-Anwendungen und deren Anwendung eingegangen.

Daraufhin werden im Kapitel \nameref{Chap:Anforderungen} die technischen und funktionalen Anforderungen beschrieben, die durch das Prototyping-Modell schrittweise verbessert und bei Bedarf erweitert wurden. Zusätzlich wird eine kurze Stakeholderanalyse durchgeführt.

...



	\chapter{Theoretische Grundlagen}\label{Chap:TheoretischeGrundlagen}
In diesem Abschnitt wird zunächst die Firma Infineon Technologies AG, in Kooperation mit der diese Arbeit entstanden ist, vorgestellt. Anschließend werden die theoretischen Grundlagen der Halbleitertechnologie geklärt - dem Hauptgeschäftsfeld der Infineon Technologies AG -, um den Nutzen der internen Software-Applikation \gls{REALIS} verständlich zu machen. Im darauffolgenden Kapitel, welches das System \gls{REALIS} näher bringen soll, wird auf dessen sogennanten Project Lifecycle, die Architektur und Statistiken eingegangen. Der technische Feasibility Check, der einen spezifischen Teil dieser Applikation darstellt, wird im letzten Teil des Abschnitts behandelt.

\section{Infineon Technologies AG}

Die Infineon Technologies AG zählt zu den weltweit führenden Herstellern von Halbleitern in den Bereichen Automotive, Power \& Sensor Systems, Green Industrial Power und Connected Secure Systems. Mit rund 58.600 Mitarbeitern ist das Unternehmen global tätig und betreibt insgesamt 84 Standorte \cite{infineon2024unternehmenspraesentation}. Einer dieser Standorte ist Regensburg mit mehr als 3000 Mitarbeitern, wo sowohl Entwicklung als auch Fertigung betrieben wird. Regensburg gilt dabei als Innovationslabor und Hightech-Fabrik, und ist der einzige Standort, an dem sowohl Frontend- als auch Backend-Produktion erfolgen. \cite{infineon2024regensburg}.

\section{Halbleitertechnologie}

Die Halbleitertechnologie ermöglicht es, elektronische Schaltungen vollständig in einem einzigen Herstellungsverfahren zu erzeugen. Dabei entstehen alle elektronischen Bauelemente und elektrischen Verbindungen auf einem monolithischen Halbleiterplättchen, das als integrierter Schaltkreis, auf Englisch \gls{IC} bezeichnet wird. Diese kleinen, dünnen Plättchen bestehen in der Regel aus Silizium und werden nach der Vereinzelung als Chips bezeichnet. Ein fertiger Wafer, aus dem diese Chips hergestellt werden, ist in Abbildung \ref{fig:Silizium-Wafer} zu sehen.

Halbleitermaterialien zeichnen sich durch ihre besondere Fähigkeit aus, elektrischen Strom nur unter bestimmten Bedingungen zu leiten. Anders als Metalle, deren Leitfähigkeit bei steigender Temperatur abnimmt, wird die Leitfähigkeit von Halbleitern mit zunehmender Temperatur exponentiell größer. Diese Eigenschaft, kombiniert mit der Möglichkeit, die Leitfähigkeit gezielt durch Dotierung – das Einbringen von Fremd\-atomen – zu steuern, macht Halbleiter so vielseitig einsetzbar.

Zur Herstellung integrierter Schaltkreise wird hochreines, monokristallines Halbleitermaterial benötigt, bei dem alle Atome in einer gleichmäßigen, durchgehenden Struktur angeordnet sind. Da solche Strukturen in der Natur nicht vorkommen, müssen sie technisch durch das ''Züchten'' von Kristallblöcken in Stangenform erzeugt werden. Diese Stangen werden in dünne Scheiben - sogenannte Wafer - geschnitten, die als Ausgangsmaterial für die Chip-Produktion dienen. Ein Wafer kann je nach Größe hunderte bis zehntausende Chips enthalten, die alle gleichzeitig hergestellt werden können \cite{lienig2023halbleitertechnologie}.

\begin{figure}[!h]
    \centering
    \includegraphics[width=0.9\textwidth]{bilder/SiC-Wafer-Infineon.jpg}
    \caption{Fertiger Silizium-Wafer mit Chips \cite{infineon2024dünnsterWafer}}
    \label{fig:Silizium-Wafer}
\end{figure}

Silizium ist das am häufigsten verwendete Material in der Halbleiterindustrie, weil es viele praktische Vorteile bietet. Es hat die perfekte Balance für den Einsatz in verschiedenen elektronischen Anwendungen und funktioniert gut bei normalen Betriebstemperaturen. Silizium bildet ein stabiles und zuverlässiges Isoliermaterial, das in Schaltkreisen vielseitig eingesetzt werden kann. Es leitet Wärme effizient ab, was wichtig ist, um eine Überhitzung zu vermeiden, besonders bei kleinen und leistungsstarken Chips. Außerdem lässt sich Silizium einfach in großen, reinen Kristallen herstellen, die für eine gleichmäßige Leistung in der Chipproduktion entscheidend sind.

Die Fertigung beginnt mit einem Rohwafer, auf dem im Front-End-of-Line (FEOL) alle Dotierungen erfolgen. Im darauf folgenden Back-End-of-Line (BEOL) werden abwechselnd isolierende und metallische Schichten aufgetragen und strukturiert, wodurch Leiterbahnen und Durchkontaktierungen entstehen. Die Strukturen moderner Chips sind dabei im Nano- bis Mikrometerbereich angesiedelt \cite{lienig2023halbleitertechnologie}.

Am Ende des Fertigungsprozesses werden die integrierten Schaltkreise (\gls{IC}), die in parallelen Reihen und Spalten auf der Oberfläche des Wafers angeordnet sind, wie in Abbildung \ref{fig:Silizium-Wafer} gut erkennbar, durch senkrecht verlaufende Schnitte voneinander getrennt. Durch diesen Schritt entstehen kleine, rechteckige, dünne Plättchen, die als Chips bekannt sind. Die Wafer, die aus den gezüchteten Kristallstäben mit Innenlochsägen ausgeschnitten werden, sind kreisrunde Scheiben und weisen typischerweise Durchmesser von 200 bis 450mm auf. Die Dicke moderner Wafer liegt heute bei knapp unter 1 mm. Fortschrittliche Technologien wie die der Infineon Technologies AG ermöglichen jedoch bereits die Fertigung und Verarbeitung ultradünner Silizium-Wafer mit einer Dicke von nur 20 Mikrometern \cite{infineon2024dünnsterWafer}.

Mit zunehmender Miniaturisierung der Halbleiterprozesse steigen die Herstellungskosten, da die Technologie komplexer wird. Jedoch sinkt durch die Verkleinerung der benötigte Platz pro Funktionseinheit auf dem Chip, was die höheren Prozesskosten kompensieren kann. So führt jede neue Chip-Generation dazu, dass mehr Leistung für den gleichen Preis erzielt wird, also eine höhere Funktionalität pro investiertem Geld \cite{lienig2023halbleitertechnologie}.
\section{REALIS}\label{Sec:REALIS}
Wird bei Infineon von einem Kunden ein neues Produkt angefordert oder entwickelt Infineon selbst ein neues Produkt, so muss dieses zunächst getestet und qualifiziert werden, bevor es in Masse produziert werden kann. Mit Produkt ist dabei ein fertiger Chip, der auf einem Wafer hergestellt wurde, gemeint. Für diese Zuverlässigkeits- bzw. Qualitäts-Tests wurde bei Infineon eine Software mit dem Namen \gls{REALIS} entwickelt. Dieses System umfasst die komplette Planung und Dokumentation der Durchführung und Ergebnisse dieser Tests. Das System beinhaltet eine gleichnamige Datenbank, in der alle wichtigen Informationen gespeichert werden.

\subsection{Projekt-Lebenszyklus}\label{Subsec:project-lifecycle}
Um ein neues Produkt zu testen, wird vom sogenannten \gls{QM} ein neues Projekt in \gls{REALIS} angelegt. Dieses befüllt er mit verschiedenen (Stress-)Tests, basierend auf vorhandenen Templates, die Arbeitsschritte (Operationen), Start- und Enddaten, Parameter der Operationen einzelner Tests und weitere Informationen enthalten. Dieser erste Schritt entspricht der obersten Zeile in Abbildung \ref{fig:realis-project-lifecycle} und bildet den Anfang eines REALIS Projekt-Lebenszyklus. 

Für jeden der folgenden Schritte wird in \gls{REALIS} der ``State``(Status) der Tests eines Projektes verändert und damit der Fortschritt dokumentiert. Dabei steht dieser zu Beginn immer auf  ``NEW`` und wird anschließend nach jedem der im Folgenden beschriebenen Schritte auf einen neuen ``State`` geändert (vgl. Abbildung \ref{fig:realis-project-lifecycle}, rechte Spalte). Welcher neue Zustand einem Test zugewiesen wird, wird dadurch entschieden, ob der beschriebene Schritt erfolgreich durchgeführt werden konnte oder nicht.

Im zweiten Schritt des Lebenszyklus weist der \gls{QM} das Projekt durch einen internen Mechanismus, einem ''State-Change'' des Projekts, einem sogenannten \gls{RPT}-Labor zu. 

In dem festgelegten \gls{RPT}-Labor validieren im Anschluss Mitarbeiter manuell die Richtigkeit der angelegten Tests und überprüfen daraufhin, ob sie die angelegten Stresstests des Projektes auch durchführen können. 
Für die Validierung der Tests werden die Stressparameter auf deren Sinnhaftigkeit überprüft. Die Frage der Durchführbarkeit hängt davon ab, ob im zugewiesenen \gls{RPT}-Labor Maschinen vorhanden sind, die in der Lage sind, die geforderten Stressoperationen auszuführen und die festgelegten Stressparameter einzuhalten. Zudem müssen diese Maschinen betriebsbereit sein – also weder in Benutzung noch außer Betrieb.

Diese Prüfungen bezeichnen den aktuellen technischen Feasibility Check, dessen Ergebnisse in \gls{REALIS} dokumentiert werden.
Falls für einige Operationen bzw. Tests keine gültigen Maschinen vorhanden sind, werden diese Tests an andere \gls{RPT}-Labore delegiert. Dadurch müssen die zu testenden Produkte jedoch von einem Labor zum anderen transportiert werden, was aufgrund der weltweiten Verteilung viel Zeit in Anspruch nehmen kann.

\begin{figure}[!h]
    \centering
    \includegraphics[width=1\textwidth]{bilder/realis-project-lifecycle.png}
    \caption{REALIS Project Lifecycle \cite{REALISWikiIntern}}
    \label{fig:realis-project-lifecycle}
\end{figure}

Anschließend wird ein ''Sample'', also eine kleine Stückzahl des Produktes, zum beauftragten \gls{RPT}-Labor geschickt. Das ''Sample'' wird in der Fachsprache auch als \textit{Test lot}, oder zu Deutsch: \textit{Test-Los} bzw. nur \gls{los} bezeichnet. Die Stückzahl wird dabei in \gls{REALIS} dokumentiert. Die  Daten der Testoperation werden dann final festgelegt und der Test-Status wird geändert.

Daraufhin erfolgt die planmäßige Durchführung der einzelnen Operationen der\linebreak Stresstests. Dabei werden der Fortschritt und die Ergebnisse von Labor-Mitarbeitern, sogenannten \glspl{operator}, in \gls{REALIS} dokumentiert. Treten während der Stresstests Probleme oder Fehler auf, werden diese an den \gls{QM} weitergeleitet, der über das weitere Vorgehen entscheidet.

Nachdem alle Tests vollständig durchgeführt und dokumentiert worden sind, muss der \gls{QM} die Ergebnisse prüfen und bewerten. Zum Schluss werden die Tests dann archiviert, wobei aus Gewährleistungsgründen die Chips und Testergebnisse 16 Jahre aufbewahrt werden müssen. Damit ist der Projekt-Lebenszyklus abgeschlossen.

\subsection{Architektur und Technologie}
Das ursprüngliche Frontend von REALIS war eine Windows-Desktop-Applikation, die sowohl vom \gls{RPT}-Labor-Mitarbeiter als auch vom \gls{QM} genutzt wurde. Im Zuge einer Modernisierung wird das System schrittweise zu einer Web-Applikation migriert. Zeitgleich erfolgt eine Aufteilung in zwei separate Anwendungen, eine für den \gls{QM} und eine für den \gls{operator}, mit dem Ziel, die Geschäftsprozesse zu vereinfachen und die Nutzerfreundlichkeit zu verbessern.

Abbildung \ref{fig:realis-komponentendiagramm} zeigt die aktuelle Systemarchitektur in Form eines Komponentendiagramms. Im Backend (grün dargestellt) kommuniziert der \texttt{REALIS-Server} über eine \texttt{DataAccess}-Schnittstelle direkt mit der zentralen \texttt{REALIS-Datenbank}, welche auf Oracle basiert. Der \texttt{REALIS-Server} stellt die Geschäftslogik (Business-Layer) bereit und wird über eine \texttt{REST-API} von den Frontends genutzt.

\begin{figure}[!h]
    \centering
    \includegraphics[width=1\textwidth]{bilder/REALIS-Komponentendiagramm.png}
    \caption{REALIS Komponentendiagramm}
    \label{fig:realis-komponentendiagramm}
\end{figure}

Das Frontend besteht aus zwei Hauptkomponenten (blau dargestellt):
\begin{enumerate}
    \item \textbf{REALIS-Desktop-Applikation:} \\
Die ursprüngliche Desktop-Anwendung, die sukzessive durch die Integration von neuen Web-Funktionalitäten modernisiert wird. Diese Web-Module, die mit dem Framework Angular entwickelt werden, sind als Subsystem (\texttt{REALIS-Web}) innerhalb der Desktop-Applikation eingebettet. Dieses System steht sowohl dem \gls{QM} als auch dem \gls{RPT}-Labor-Mitarbeiter zur Verfügung.

\item \textbf{REALIS-Web-Operator-System:} \\
Eine eigenständige Web-Applikation, die speziell für die Anforderungen des \gls{RPT}-Labors entwickelt wird. Diese Anwendung befindet sich noch in der Entwicklung, wird jedoch bereits für einige Aufgaben eingesetzt. Für nicht implementierte Funktionen muss der \gls{RPT}-Mitarbeiter vorübergehend auf die alte Desktop-Applikation ausweichen. Zusätzlich ist geplant, das \texttt{REALIS-Web-Operator-System} als native iOS-App für mobile Apple-Geräte (z. B. iPads, iPhones) bereitzustellen. Die Verwendung von Angular ermöglicht dabei eine plattformübergreifende Entwicklung, die sowohl als Web-App als auch als native App funktioniert.
\end{enumerate}

Das Diagramm \ref{fig:realis-komponentendiagramm} verdeutlicht die Trennung zwischen Backend und Frontend sowie die unterschiedlichen Nutzerrollen (\gls{QM} und \gls{RPT}-Employee), die spezifische Zugriffsrechte auf die jeweiligen Systeme haben.


\subsection{Weitere Funktionen und Statistiken}
Neben der Möglichkeit, Qualitätstests (Reliability-Tests) anzulegen, bietet \gls{REALIS} eine Vielzahl zusätzlicher Funktionen, die dazu beitragen, Prozesse effizienter zu gestalten und Engpässe zu vermeiden. So unterstützt das System beispielsweise die Planung individueller Laborkapazitäten, wodurch unnötige Investitionen vermieden und vorhandene Ressourcen optimal genutzt werden können. Darüber hinaus ermöglicht \gls{REALIS} die Referenzierung bereits durchgeführter Testergebnisse, um redundante Tests zu vermeiden und Zeit sowie Kosten zu sparen.

Nach Abschluss eines Tests können in \gls{REALIS} automatisch benötigte Ergebnisberichte generiert werden – sowohl für den Kunden als auch für das \gls{RPT}-Labor. Dies erleichtert die Dokumentation und erhöht die Effizienz im Testmanagement.

Seit 2001 ist \gls{REALIS} im Einsatz und verzeichnet derzeit etwa 4.300 aktive Nutzer. Das System findet Anwendung in 101 \gls{RPT}-Laboren in 17 Ländern. Aktuell verwaltet \gls{REALIS} rund 270.000 Projekte mit etwa 1,9 Millionen Stresstests \cite{REALISPowerPointIntern}.
\section{Technischer Feasibility Check}
Der automatisierte technische Feasibility Check, bezeichnet eine ''Machbarkeitsprüfung'' oder ''Machbarkeitsstudie'', die bewertet, ob ein Stresstest an einem \gls{RPT}-Laborstandort durchgeführt werden kann.

Der Check ist ein integraler Bestandteil des \gls{REALIS}-Projekt-Lifecycles, wie in Kapitel \ref{Subsec:project-lifecycle} beschrieben. Derzeit wird dieser manuell von Mitarbeitern des \gls{RPT}-Labors durchgeführt und in \gls{REALIS} dokumentiert. Dafür prüfen sie, ob die im Labor vorhandenen Kapazitäten und Bedingungen ausreichen, um die geforderten Tests für das jeweilige Produkt durchzuführen.

Die Beurteilung basiert auf mehreren Parametern. Um diese besser zu verstehen, wird im folgenden Kapitel erläutert, wie ein Stresstest im Detail abläuft und welche Schritte dabei erforderlich sind.


\subsection{Stresstests (Reliability Tests)}

Stresstests, auch als \textit{Reliability Tests} bezeichnet, sind ein wesentlicher Bestandteil der Qualifikation neuer Produkte. Sie dienen dazu, die Belastbarkeit und Zuverlässigkeit eines Produkts unter definierten Bedingungen zu überprüfen. Zu den häufigsten Testarten zählen Temperaturtests (Hitze- und Kältetests), Drucktests, Feuchtigkeitstests und elektrische Tests – häufig auch in Kombination.

Jeder Stresstest setzt sich aus mehreren aufeinanderfolgenden Operationen zusammen, die von einem \gls{RPT}-Labor-\gls{operator} mithilfe eines Test-\glspl{lot} (einer kleinen Stückzahl des Produkts) durchgeführt werden. Üblicherweise beginnt ein Test mit einer ''START''-Operation und endet mit einer ''END''-Operation. Dazwischen erfolgen mehrere Stressoperationen, die den spezifischen Stresstyp des Tests abbilden und die eigentliche Belastungsprüfung darstellen. Nach jeder Stressoperation wird das \gls{lot} einer Funktionsprüfung (weitere Operation) unterzogen, um sicherzustellen, dass alle Chips die vorhergehende Belastung unbeschadet überstanden haben.

Während der Durchführung der Stressoperationen platziert der \gls{operator} das \gls{lot} in eine geeignete Testmaschine und passt die Parameter gemäß den geplanten Vorgaben in \gls{REALIS} an. Dabei kann eine Operation mehrere verschiedene Parameter enthalten, mit vordefinierten Werten. Beispielsweise wird bei Temperaturtests die Zieltemperatur sowie die Testdauer eingestellt, während bei anderen Tests, wie etwa mechanischen Belastungen, die Anzahl der Wiederholungen definiert wird. Die Chips des \glspl{lot} werden dabei häufig auf einem sogenannten \gls{board} montiert, das als Träger fungiert und Leiterbahnen enthält, um die elektrische Verbindung zwischen dem Testsystem und den Chips herzustellen.

Für kleinere Chips kommt oft ein zusätzliches Zwischenelement, das sogenannte \gls{substrate}, zum Einsatz. Das \gls{substrate} fungiert als Trägerelement und enthält ebenfalls Leiterbahnen. Es dient dazu, die Kontakte des Chips zu „vergrößern“ und somit die Handhabung und Verbindung mit dem \gls{board} zu erleichtern.

In einigen Testverfahren, wie beispielsweise in Regensburg, wird jedoch kein \gls{board} benötigt. Hier werden die Chips in eine Karte (vergleichbar mit einer EC-Karte) integriert, die selbst als \gls{substrate} dient. Die Karte wird in einer Testmaschine wiederholt mechanischen Belastungen, wie Biegungen, ausgesetzt. Die Karten werden dabei einzeln in die Maschine eingelegt, was den Einsatz eines \glspl{board} überflüssig macht.

Alle durchgeführten Operationen eines Tests müssen vom \gls{operator} in \gls{REALIS} dokumentiert werden, insbesondere die Stressoperationen und die anschließenden Funktionsprüfungen. Dabei werden unter anderem die Anzahl der Chips, die nach einer Stressoperation ausgefallen sind, sowie die vermuteten Ursachen des Ausfalls festgehalten.
Nur wenn das gesamte \gls{lot} unbeschadet ist kann mit der nächsten Operation fortgefahren werden, ansonsten wir der Test vorerst gestoppt und der \gls{QM} muss entscheiden, wie weiter vorgegangen wird.

\subsection{Parameter des technischen Feasibility Checks}\label{Subsec:ParameterdestechnischenFeasibilityChecks}

Im Rahmen des technischen Feasibility Checks wird zunächst geprüft, ob die vom \gls{QM} definierten Parameter und Werte der geplanten Stressoperationen sinnvoll und durchführbar sind. Dies umfasst beispielsweise die Überprüfung, ob die vorgegebene Testtemperatur im technisch möglichen Bereich liegt. Zudem wird kontrolliert, ob für jeden Parameter ein geeigneter Wert festgelegt wurde, falls dies erforderlich ist.

Anschließend wird geprüft, ob am zuständigen \gls{RPT}-Labor geeignete Stressmaschinen für den geforderten Stresstesttyp verfügbar sind. Dabei wird sichergestellt, dass die Maschinen einsatzbereit, nicht defekt und auch nicht in Wartung sind. Außerdem wird überprüft, ob sie die vorgegebenen Parameter und Werte tatsächlich umsetzen können. Zusätzlich wird kontrolliert, ob passende \glspl{board} und \glspl{substrate} für die Chips des \glspl{lot} vorhanden sind, sofern diese benötigt werden.

Aktuell muss ein Mitarbeiter des Labors alle diese Prüfungen für jeden Stresstest manuell durchführen. Bei wiederkehrenden Tests kann es sein, dass der Mitarbeiter bereits aus Erfahrung weiß, dass der Test durchführbar ist, und der Feasibility Check somit direkt im System bestätigt werden kann. In anderen Fällen kann es jedoch mehrere Stunden dauern, um die Machbarkeit eines Tests festzustellen.

Um diesen Prozess zu beschleunigen und eine höhere Effizienz zu erreichen, soll dieser Schritt des \nameref{Subsec:project-lifecycle} von \gls{REALIS} automatisiert werden. Durch diese Automatisierung wird nicht nur die Dauer des Feasibility Checks reduziert, sondern auch eine fundiertere Begründung für die Entscheidung ermöglicht, da das System aktuell nur eine einfache „Ja/Nein“-Auswahl zulässt. Darüber hinaus trägt die Automatisierung dazu bei, menschliche Fehler zu minimieren und die Genauigkeit der Entscheidungen zu erhöhen.
	\chapter{Anforderungen}\label{Chap:Anforderungen}

In diesem Kapitel erfolgt eine strukturierte Darstellung der Anforderungen an den automatisierten technischen Feasibility Check. Dabei werden die funktionalen Anforderungen durch nicht-funktionale Anforderungen ergänzt, die Aspekte wie Performance und Sicherheit umfassen.

Anschließend wird eine Stakeholderanalyse durchgeführt, um die beteiligten Personengruppen am Projekt zu identifizieren und um deren Interessen, Bedürfnisse und Einflüsse zu verstehen. 

\section{Funktionale Anforderungen}

\setlength{\leftskip}{1em} 
\textbf{Automatisierung des technischen Feasibility Checks}  \\
Der Feasibility Check soll vollständig automatisiert ablaufen, sodass Benutzer keine technischen Vorkenntnisse benötigen. Die Ausführung wird ausschließlich durch den Benutzer initiiert oder angefordert.

\textbf{Nachvollziehbarkeit}  \\
Um das Ergebnis des automatisierten Feasibility Checks transparent zu gestalten, müssen alle zugrunde liegenden logischen Entscheidungen in der Datenbank gespeichert werden. Diese Informationen sollen für den Benutzer in verständlicher Form dargestellt werden, insbesondere wenn die Logik auf unerwartete Probleme stößt.

\textbf{Modularer Aufbau}  \\
Der Feasibility Check besteht aus mehreren separaten ''Checks''. Diese Module, abhängig von den Parametern (siehe Kapitel~\ref{Subsec:ParameterdestechnischenFeasibilityChecks}), sollen von Systemadministratoren individuell aktiviert oder deaktiviert werden können. Ist ein Modul deaktiviert, soll der Benutzer weiterhin die Möglichkeit haben, den Check manuell durchzuführen.

\textbf{Flexibilität und Erweiterbarkeit}  \\
Das System soll so konzipiert sein, dass neue Module bzw. ''Checks'' flexibel hinzugefügt werden können, um zukünftige Anforderungen ohne größeren Aufwand zu erfüllen.

\textbf{Manuelle Durchführung weiterhin Ermöglichen}  \\
Falls der automatisierte Feasibility Check zu einem negativen Ergebnis kommt, auch aufgrund von Problemen in der bestehenden Logik, soll der Benutzer die Möglichkeit haben, den Check manuell durchzuführen.

\textbf{Integration in bestehende Architektur}  \\
Die Systemarchitektur des Feasibility Checks muss vollständig kompatibel mit der bestehenden Architektur von \gls{REALIS} sein, um eine nahtlose Integration zu gewährleisten.

\setlength{\leftskip}{0em} 

\section{Nicht-funktionale Anforderungen}
Performance, Benutzerfreundlichkeit, Sicherheit etc.


Der Feasibility Check soll persistent sein. Deswegen soll das Ergebnis und die logischen Entscheidungen verknüpft mit dem zugehörigen Stresstest fest in der Datenbank gespeichert werden. Außerdem soll während eines Checks kein anderes System oder anderer User Änderungen an diesem Test vornehmen können.

Um den Programmcode und die Logik des Feasibility Checks gründlich zu testen, soll dieser erst auf der Testversion des Systems eingeführt werden. Nach ausführlichem Testen, kann dieser anschließend auf die produktive Version ausgerollt werden.

Damit der Feasibility Check auch in Zukunft funktioniert, sollen für diesen Unittests erstellt werden, um so zu garantieren, dass dieser bei Änderungen in anderen Teilen des Systems, weiterhin gewünschte Ergebnisse liefert.

Der Feasibility Check hat keine zeitlichen Vorgaben, soll aber aufgrund der geschätzten längeren Ausführungsdauer für den User im Hintergrund ablaufen.


\section{Stakeholder-Analyse}
Im Rahmen der Entwicklung des automatisierten technischen Feasibility Checks erfolgte mehrfach wöchentlich ein Austausch mit den Stakeholdern. Diese setzen sich aus vier Schlüsselpersonen sowie einer Personengruppe zusammen.

Zu den Einzelpersonen gehört der Product Owner von \gls{REALIS}, der Application Owner, der Scrum Master sowie ein LabAdmin des \gls{RPT}-Labors in Regensburg. Ergänzt wird diese Gruppe durch das externe Entwicklungsteam von \gls{REALIS}, dessen Mitglieder in Portugal ansässig sind.

Der \textbf{Product Owner} von \gls{REALIS} vertritt die Interessen und Anforderungen aller Stakeholder, darunter die \gls{RPT}-Labore und \glspl{QM}, um ein gemeinsam abgestimmtes Produktziel zu entwickeln. Seine Hauptaufgabe besteht darin, diese Anforderungen zu priorisieren und in klar formulierte Einträge des Product Backlogs für das Entwicklungsteam zu überführen. Dabei stellt er sicher, dass das Team kontinuierlich an den wichtigsten Aufgaben arbeitet \cite{scrumguide2020}.

Der \textbf{Application Owner} ist für die strategische Ausrichtung und langfristige Verwaltung von \gls{REALIS} verantwortlich. In seiner Rolle als Applikations-Architekt sorgt er dafür, dass die Architektur-Richtlinien eingehalten werden. Zu seinen Aufgaben gehört die kontinuierliche Optimierung der Anwendung, die Sicherstellung ihrer Stabilität und Sicherheit sowie die Planung von Systemaktualisierungen, wie etwa der Wechsel vom Desktop-Frontend zu einer webbasierten Anwendung, um den geschäftlichen Anforderungen gerecht zu werden.

Der \textbf{Scrum Master} moderiert die agilen Prozesse und achtet darauf, dass diese korrekt eingehalten werden. Er unterstützt das Team dabei, sich kontinuierlich zu verbessern und effizienter zu arbeiten. Zudem fördert er die Zusammenarbeit zwischen den Teammitgliedern und sorgt für die Schaffung eines produktiven Arbeitsumfelds. Hierfür beseitigt er Hindernissen zwischen den Stakeholdern und dem Scrum Team \cite{scrumguide2020}.

Der \textbf{LabAdmin} im \gls{RPT}-Labor in Regensburg unterstützt die \gls{RPT}-Labor-\glspl{operator} bei ihrer Arbeit und hilft bei der Lösung von Problemen im Umgang mit \gls{REALIS}. Darüber hinaus entwickelt er Projekt-Test-Templates in \gls{REALIS}, die \glspl{QM} beim Anlegen von Tests in Projekten unterstützen. Er ist zudem aktiv in das Testen neuer Software-Versionen von \gls{REALIS} eingebunden.


Das \textbf{externe Entwicklungsteam} von \gls{REALIS} mit Sitz in Portugal stellt sicher, dass die Coding-Standards eingehalten werden. Es legt besonderen Wert auf eine sichere und gut dokumentierte Programmierung, um die Wartbarkeit und Weiterentwicklung des Codes zu gewährleisten.






	\chapter{Systemdesign}

\section{Architekturüberblick}
Gesamtsicht auf die Architektur der Lösung (Backend, Frontend, Datenbank).

\begin{figure}[!h]
    \centering
    \includegraphics[width=1\textwidth]{bilder/REALIS-Komponentendiagramm.png}
    \caption{REALIS Komponentendiagramm}
    \label{fig:realis-komponentendiagramm}
\end{figure}

Da der Feasibility Check als Funktionalität in die bestehende Software-Applikation \gls{REALIS} integriert werden soll, ist es notwendig, die bereits verwendeten Technologien und Programmiersprachen zu übernehmen, um eine nahtlose Integration zu gewährleisten.

Eine Evaluierung alternativer Technologien, die möglicherweise besser für die Applikation geeignet wären, wurde daher nicht durchgeführt. Stattdessen orientiert sich die Umsetzung strikt am bestehenden System.

Für das Datenbankdesign kommt eine Oracle-Datenbank mit SQL zum Einsatz. Die Backend-Logik wird in der objektorientierten Programmiersprache C\# implementiert. Für das Frontend wird das Webapplikationsframework Angular verwendet, das auf TypeScript basiert. Angular ermöglicht eine parallele Entwicklung sowohl für Webanwendungen als auch für native Mobilapplikationen.

\section{Datenbankdesign}
Struktur der Oracle-Datenbank, wichtige Tabellen und Beziehungen.

\begin{figure}[!h]
    \centering
    \makebox[\textwidth]{\includegraphics[width=0.98\paperwidth]{bilder/REALIS-Datenbankmodell.png}}
    \caption{REALIS Datenbankdesign}
    \label{fig:realis-datenbankdesign}
\end{figure}
\section{Backend-Logik}
Beschreibung der C\#-Implementierung, inklusive wichtiger Klassen und Methoden.

\begin{figure}[!h]
    \centering
    \makebox[\textwidth]{\includegraphics[width=0.85\paperwidth]{bilder/flowchart-feasibilitycheck-without-param-check.png}}
    \caption{Flowchart Feasibility Check - Condition Check }
    \label{fig:feasibility-check-condition-check}
\end{figure}

\section{Frontend-Design}
Überblick über die Angular-Anwendung, Struktur und Benutzeroberfläche.

	\chapter{Implementierung}\label{Chap:Implementierung}

\section{Projektplan für den Feasibility Check}\todo{Kapitel am anfang von Systemdesign tun}
Für die Entwicklung des automatisierten technischen Feasibility Checks wurde zu Beginn ein Projektplan in Form eines GANTT-Diagramms erstellt, der in Abbildung \ref{fig:roadmap} dargestellt ist. Dieser Plan ist in mehrere Swimlanes unterteilt und umfasst die Erarbeitung der Anforderungen, den Entwurf des Datenbankdesigns, die Backend-Entwicklung, die Frontend-Entwicklung sowie das Testen des Systems. Zusätzlich wurden für die Benutzer spezifische Meilensteine definiert, die in der unteren Zeile des Diagramms veranschaulicht werden.

\begin{figure}[!htbp]
    \centering
    \includegraphics[width=1\textwidth]{bilder/Roadmap.pdf}
    \caption{Feasibility Check Projektplan}
    \label{fig:roadmap}
\end{figure}

Die geplanten Maßnahmen für die ersten zwei Monate konnten weitgehend umgesetzt werden, mit Ausnahme der Endanwender-Tests. In den darauffolgenden Monaten lag der Schwerpunkt auf der Überarbeitung bestehender Algorithmen im Backend, da wiederholt neue Ausnahmefälle identifiziert wurden. Auch das Frontend wurde mehrfach angepasst, um den aktuellen Anforderungen gerecht zu werden. Dadurch konnten alle optionale Meilensteine, sowie die Implementierung weiterer Checks inklusive des er \gls{substrate} Checks, nicht realisiert werden – unter anderem aufgrund einer zu optimistischen Zeiteinteilung.

Aktuell sind das Datenbankdesign und das Backend bereits vom Testing-System auf das Staging-System ausgerollt worden und werden dort von einzelnen Anwendern getestet. Für das Backend wurden zudem Unittests entwickelt, was im Frontend leider nicht mehr möglich war. Weitere geplante Meilensteine konnten letztlich aufgrund von Zeitbeschränkungen nicht umgesetzt werden.


\section{Entwicklungsumgebung}
Tools und Technologien, die verwendet wurden (IDE, Frameworks, Datenbank-Tools).
Angular, Visual Studio Code, 
C\#, Visual Studio,
Oracle, Pl sql developer

\section{Implementierung des Feasibility Checks}
Detaillierte Beschreibung des Implementierungsprozesses.
(Beschreibung der C\#-Implementierung, inklusive wichtiger Klassen und Methoden.
flussdiagramme
nhibernate, try catch blöcke,)
\section{Integration}
Wie die verschiedenen Teile (Frontend, Backend, Datenbank) miteinander interagieren.

\begin{figure}[!h]
    \centering
    \includegraphics[width=1\textwidth]{bilder/Sequence-Integration.png}
    \caption{Sequenz-Diagramm FeasibilityCheck}
    \label{fig:sequence-diagram}
\end{figure}

	\chapter{Test und Evaluation}

Bisher wurde der Feasibility Check nur wenig von tatsächlichen Endanwendern genutzt; sämtliche bisher durchgeführten Tests beziehen sich ausschließlich auf Unittests der Backend-Logik. Diese Unittests wurden mit großem Aufwand und hoher Sorgfalt entwickelt, um sicherzustellen, dass die bestehende Logik auch in Zukunft stabil bleibt und Änderungen keine unerwarteten Fehler verursachen.

\section{Unittests}

Die Entwicklung der Unittests für den Feasibility Check gestaltete sich als besonders anspruchsvoll. Dies liegt vor allem daran, dass es nicht ausreicht, der Methode \texttt{FeasibilityCheck()} einfache fehlerhafte Eingaben zu übermitteln. Vielmehr erwartet diese Methode eine existierende Test-ID, die in der Datenbank vorhanden ist und im Regelfall mit Operationen sowie zugehörigen Parametern befüllt sein muss. Zusätzlich hängt das Ergebnis des Feasibility Checks von mehreren Faktoren ab, wie etwa der Feasibility-Konfiguration, den einzelnen Condition- und Equipment-Checks, der \texttt{op\_data\_param\_type}-Tabelle sowie den in der Datenbank hinterlegten Maschinen.

Aufgrund dieser Komplexität ist es notwendig, gezielt ''Fake-Daten'' in einer Testdatenbank anzulegen. Aufbauend auf diesen Testdaten wird der Feasibility Check ausgeführt und das Ergebnis hinsichtlich Korrektheit und Vollständigkeit validiert. Nach erfolgreicher Überprüfung erfolgt eine sichere Entfernung der ''Fake-Daten'', um die Integrität der Datenbank zu gewährleisten.

Um diesen Prozess zu vereinfachen, wurde die Klasse \texttt{FeasibilityCheckScenario} eingeführt, die auf einer Projektszenario-Klasse basiert. Dieses Konzept stellt eine Sammlung einfacher Methoden bereit, mit denen entweder ein standardisierter Feasibility-Test oder ein spezifischer Test mit vorgegebenen Inhalten angelegt werden kann. Über public Boolean-Parameter in der \texttt{FeasibilityCheckScenario}-Klasse können verschiedene Szenarien aktiviert werden, die intern zur Erzeugung der entsprechenden Fake-Daten führen. Beispielsweise kann das Szenario \texttt{PlanValueOutOfRange} aktiviert werden, um einen Testfall zu simulieren, bei dem der PlanValue außerhalb des erlaubten Bereichs liegt. Ein weiteres Beispiel ist das Szenario \texttt{WrongMachineDate}, bei dem eine Maschine, die normalerweise für den Test zugelassen wäre, mit einem inkorrekten Datum erstellt wird. Standardmäßig sind alle Szenarien deaktiviert und können je nach Bedarf einzeln oder in Kombination aktiviert werden. Abbildung~\ref{fig:unittests-parameters} veranschaulicht die möglichen Boolean-Parameter, die als Szenarien innerhalb der \texttt{FeasibilityCheckScenario}-Klasse dienen.

\begin{figure}[!htb]
    \centering
    \includegraphics[width=1\textwidth]{bilder/unittests-parameters.png}
    \caption{Boolean-Parameter der \texttt{FeasibilityCheckScenario}-Klasse}
    \label{fig:unittests-parameters}
\end{figure}

Durch diese abstrahierte Vorgehensweise lassen sich die umfangreichen Unittests strukturiert und effizient umsetzen. Zwei Beispiel-Unittests sind in Abbildung~\ref{fig:unittestcases} dargestellt. Hierbei legt die Methode \texttt{CreateDefaultFeasibilityProject()} einen Default-Test an, der jeweils mit einer Operation und einem Parameter befüllt ist, um das Testszenario zu vereinfachen. Anschließend generiert die Methode \texttt{CreateMockData()} in der Datenbank alle erforderlichen ''Fake-Daten'' bzw. passt bestehende Datensätze an, beispielsweise in der Feasibility-Konfigurationstabelle. Die Erstellung dieser ''Fake-Daten'' erfolgt abhängig von den eingestellten Boolean-Parametern, sodass die entsprechenden Szenarien erfüllt werden. Im zweiten Unittest, wie in Abbildung~\ref{fig:unittestcases} gezeigt, wurde beispielsweise das Szenario \texttt{WrongMachineDate} aktiviert. Danach wird über die Methode \texttt{PerformFeasibilityCheck()} der Feasibility Check ausgeführt, wobei das zurückgelieferte Ergebnis abschließend evaluiert wird. Ein automatisierter Cleanup-Mechanismus sorgt schließlich dafür, dass die initial angelegten Fake-Daten sicher wieder aus der Datenbank entfernt werden.

\begin{figure}[!htb]
    \centering
    \includegraphics[width=1\textwidth]{bilder/unittestcases.png}
    \caption{Unittests für den Feasibility Check}
    \label{fig:unittestcases}
\end{figure}

Die Unittests sind in ein automatisiertes Test-Framework integriert, sodass bei jeder Änderung der Backend-Logik automatisch überprüft wird, ob alle Testfälle weiterhin erfolgreich durchlaufen werden. Diese Vorgehensweise unterstützt die kontinuierliche Qualitätssicherung und trägt maßgeblich zur Stabilität und Wartbarkeit des Systems bei.

\section{Ergebnisse der Tests}
Darstellung der Testergebnisse und ihrer Bedeutung.

\section{Evaluation des Systems}
Bewertung der Lösung hinsichtlich der definierten Anforderungen.

Die implementierten Unittests bilden somit eine solide Grundlage für die Qualitätssicherung des Feasibility Checks. Dennoch bleibt die Integration des gesamten Systems unter realen Betriebsbedingungen ein zukünftiges Ziel. In diesem Zusammenhang sind weitere Testmethoden wie Integrationstests und Usability-Tests vorgesehen, um sowohl die technische Robustheit als auch die Anwenderfreundlichkeit des Systems umfassend zu evaluieren.

	\chapter{Diskussion}
In diesem Kapitel werden die zentralen Aspekte der Systementwicklung reflektiert und diskutiert. Dabei werden die Herausforderungen, die während des Projekts aufgetreten sind, analysiert und die daraus gewonnenen Lernerfahrungen zusammengefasst. Abschließend wird ein Ausblick auf mögliche Weiterentwicklungen und Optimierungen des Systems gegeben, um dessen zukünftige Einsatzfähigkeit und Effizienz zu steigern. Die Diskussion dient dazu, die gemachten Erfahrungen zu strukturieren und Potenziale für zukünftige Arbeiten aufzuzeigen.
\section{Herausforderungen}

Während der Entwicklung des Systems traten mehrere Herausforderungen auf, die bewältigt werden mussten. Eine der ersten und zentralen Aufgaben war die Ausarbeitung der Anforderungen (Requirements). Dies erforderte eine präzise Kommunikation mit den Stakeholdern, um sicherzustellen, dass alle funktionalen und nicht-funktionalen Anforderungen eindeutig definiert und umfassend dokumentiert wurden. 

Ein weiteres Hindernis bestand darin, sich in das umfangreiche und komplexe Datenbankmodell von \gls{REALIS} einzuarbeiten. Die Größe und Komplexität des Modells machten es notwendig, sich intensiv mit der Struktur und den Zusammenhängen der Daten auseinanderzusetzen, um die für den Feasibility Check relevanten Tabellen zu identifizieren. Nur durch dieses tiefgehende Verständnis konnte garantiert werden, dass die Erweiterungen effizient und korrekt in das System integriert werden konnten.

Zu den Herausforderungen zählte auch die Entwicklung eines stabilen Algorithmus, der in allen Szenarien zuverlässig funktioniert und kein unerwartetes Verhalten zeigt. Dabei war es wichtig, dass der Feasibility Check nicht nur korrekte Ergebnisse liefert, sondern auch sinnvolle Fehlermeldungen generiert, um potenzielle Probleme frühzeitig zu erkennen und zu behandeln. Dies erforderte eine sorgfältige Planung, Implementierung und umfangreiche Tests, um die Robustheit des Systems zu gewährleisten.

\section{Lernerfahrungen}

Das Projekt bot zahlreiche Gelegenheiten, wertvolle Lernerfahrungen zu sammeln. Die Programmierkenntnisse in C\# und Angular sowie im Bereich des Datenbankdesigns konnten deutlich vertieft werden. Besonders die praktische Anwendung dieser Technologien in einem realen Projektkontext trug dazu bei, ein besseres Verständnis für Softwareentwicklung und deren Herausforderungen zu entwickeln.

Ein weiterer wichtiger Aspekt war die Erkenntnis, wie essenziell eine ausführliche Dokumentation ist. Sie dient nicht nur als Referenz für das Entwicklungsteam, sondern erleichtert auch die Kommunikation mit Stakeholdern und die spätere Wartung des Systems. Zudem wurde deutlich, wie wichtig es ist, frühzeitig die Anforderungen klar zu definieren und zu dokumentieren. Gleichzeitig ist es hilfreich, bereits in frühen Phasen Prototypen in Form von Diagrammen oder ''Mock-ups'' zu erstellen, um das gemeinsame Verständnis im Team zu fördern und Missverständnisse zu vermeiden.

Ein weiterer zentraler Lernpunkt war die Bedeutung von sicherem und stabilem Code. Durch die Implementierung von robusten Fehlerbehandlungsmechanismen und die Erstellung von qualitativ hochwertigen Unittests konnte die Zuverlässigkeit des Systems erheblich erhöht werden.  Darüber hinaus zeigte sich, wie wichtig eine gute Zusammenarbeit im Team ist. Die regelmäßige Abstimmung, das Teilen von Wissen und die gemeinsame Lösung von Problemen waren entscheidend für den Erfolg des Projekts.

\section{Ausblick}

Das System bietet zahlreiche Möglichkeiten für Weiterentwicklungen und Ver-\linebreak besserungen. Eine potenzielle Erweiterung wäre die Implementierung zusätzlicher Checks, wie beispielsweise eines Substrat Checks oder eines Board Checks, um die Funktionalität des Systems zu vervollständigen. Zudem sollten die bestehenden Unittest-Fälle erweitert werden, um eine noch umfassendere Abdeckung der Codebasis zu gewährleisten und die Stabilität des Systems weiter zu steigern.

Ein weiterer wichtiger Schritt wäre die ausführliche Durchführung von Tests mit tatsächlichen Endnutzern. Solche Tests könnten wertvolle Einblicke in die Benutzerfreundlichkeit und die praktische Anwendbarkeit des Systems liefern. Zudem wäre eine Zeitanalyse sinnvoll, um zu quantifizieren, wie viel Zeit durch die Automatisierung im Vergleich zur manuellen Bearbeitung eingespart werden kann. Dies würde den Nutzen des automatisierten Systems deutlich unterstreichen.

Auch die Erweiterung des Frontends bietet Potenzial. So könnte eine Benutzeroberfläche implementiert werden, die das Einfügen von Min- und Max-Werten bzw. Parametern für Equipment- und Condition Checks sowie die Konfiguration der Feasibility Checks ermöglicht. Zudem wäre es sinnvoll, eine Funktion zu entwickeln, die es den Nutzern erlaubt, nach Durchführung des Feasibility Checks aus den als geeignet bewerteten Maschinen eine endgültige Auswahl zu treffen. Dieser Auswahlprozess könnte durch Machine-Learning-Algorithmen unterstützt werden, um die optimale Maschine basierend auf historischen Daten und spezifischen Kriterien vorzuschlagen. Solche Erweiterungen würden die Flexibilität und Benutzerfreundlichkeit des Systems weiter steigern und seine Einsatzmöglichkeiten signifikant erweitern.


	\chapter{Fazit}\label{Chap:Fazit}

\chapter{Anhang}
\section{Code-Snippets}
\section{Dokumentation}
\section{Weitere Materialien}

	
	% Literaturverzeichnis in das Inhaltsverzeichnis einfügen
	\addcontentsline{toc}{chapter}{Literaturverzeichnis}
	
	% Style für die Bibliothek festlegen
  	\bibliographystyle{IEEEtran}
  	
  	% Einfügen des Literaturverzeichnisses in das Dokument
	\bibliography{references}
	
	\newpage
		
	\thispagestyle{empty}	
	\addcontentsline{toc}{chapter}{Abkürzungsverzeichnis}\label{Sec:Abkuerzungen}
	\chapter*{Abkürzungsverzeichnis}
	\begin{acronym}[OTH R]
	 \acro{OTH R}{Ostbayerische Technische Hochschule Regensburg}
	 \acro{OTH}{Ostbayerische Technische Hochschule}
	 \acro{GCC}{Der Compiler namens gcc}
	 \acro{I2C}[I²C]{Inter-Integrated Circuit}
	 \acro{REALIS}{Reliability evaluation and logistic information system}
	\end{acronym}
	
	% Anhang
	\addcontentsline{toc}{chapter}{Abbildungsverzeichnis}
	\listoffigures
	
	\addcontentsline{toc}{chapter}{Tabellenverzeichnis}
	\listoftables
	
	%\addcontentsline{toc}{chapter}{Listingsverzeichnis} % für Quellcode
	%\lstlistoflistings 
	
	%\addcontentsline{toc}{chapter}{Digitaler Anhang}	% für digitalen Anhang, falls nötig
	%\include{anhang}
	\cleardoublepage
	\thispagestyle{empty}
	\makedeclaration

	\cleardoublepage

\end{document}
