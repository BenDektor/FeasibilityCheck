\newglossaryentry{operator}
{
    name=Operator,
    description={RPT-Labor Mitarbeiter, zuständig für Validierung, Durchführung und Dokumentation von Stresstests}
}
\newglossaryentry{los}
{
    name=Los,
    plural=Lose,
    description={Auch als Test lot bzw. Test-Los Kleine Stückzahl von produzierten Chips die Stresstests unterzogen werden.}
}
\newglossaryentry{board}
{
    name=Board,
    description={Behälter für Chips eines \glspl{los}s}
}
\newglossaryentry{substrate}
{
    name=Substrat,
    plural=Substrate,
    description={Trägersubstanz für einen Chip eines \glspl{los}s}
}
\newglossaryentrywithacronym{QM}{Quality Manager}{
    Besitzer und Verantwortlicher eines REALIS-Projects, legt dieses an.
}

\newglossaryentrywithacronym{REALIS}{Reliability Evaluation and Logistic Information System}{
    Software-Applikation zur Qualifizierung neuer Halbleiter-Produkte
}
\newglossaryentry{ConditionCheck}
{
    name=Condition Check,
    description={Teil des Feasibility Checks, bei dem überprüft wird, ob der geplante Wert in einem sinnvollen Bereich liegt.}
}
\newglossaryentry{EquipmentCheck}
{
    name=Equipment Check,
    description={Teil des Feasibility Checks, bei dem überprüft wird, ob der geplante (Stress-)Test durchgeführt werden kann. Also ob für die geforderten Parameter, eine Maschine zur Verfügung steht.}
}


% Define acronyms

\newacronym{IC}{IC}{Integrated Circuit}
\newacronym{FEOL}{FEOL}{Front-End-of-Line}
\newacronym{BEOL}{BEOL}{Back-End-of-Line}
\newacronym{RPT}{RPT}{Reliability Product Testing}