\chapter{Fazit}

Die Entwicklung des Feasibility-Check-Systems war ein äußerst lehrreiches und erfolgreiches Projekt, das nicht nur meine fachlichen Fähigkeiten erweitert, sondern auch wertvolle Einblicke in die Zusammenarbeit im Team und die praktische Softwareentwicklung geboten hat. Die enge und produktive Zusammenarbeit im Team war ein zentraler Erfolgsfaktor. Durch regelmäßige Abstimmungen, offene Kommunikation und den Austausch von Wissen konnten wir Herausforderungen effizient meistern und gemeinsam Lösungen erarbeiten. Diese Erfahrung hat mir gezeigt, wie wichtig eine gute Teamdynamik und klare Rollenverteilung für den Projekterfolg sind.

Ein besonderer Erfolg des Projekts ist die Tatsache, dass der entwickelte Feasibility Check nicht nur theoretisch ausgearbeitet, sondern auch praktisch umgesetzt wurde. Das System befindet sich bereits in der ''Staging-Phase'' und wird im nächsten Schritt in den produktiven Einsatz überführt. Dort wird es von Endnutzern ausführlich getestet und genutzt werden können. Dies unterstreicht den praktischen Nutzen des Systems und zeigt, dass es einen echten Mehrwert für die Anwender bietet.

Neben den technischen Aspekten habe ich auch viel über die Bedeutung von strukturierter Planung, Dokumentation und kontinuierlicher Qualitätssicherung gelernt. Die Erstellung von klaren Anforderungen, die Entwicklung von robustem Code und die Implementierung von umfangreichen Unittests waren entscheidend, um ein zuverlässiges und fehlerfreies System zu schaffen. Diese Erfahrungen werden mir auch in zukünftigen Projekten von großem Nutzen sein.

Zusammenfassend lässt sich sagen, dass das Projekt nicht nur die gestellten Anforderungen erfüllt hat, sondern auch eine solide Grundlage für zukünftige Erweiterungen und Optimierungen bietet. Die positiven Rückmeldungen und die bevorstehende produktive Nutzung des Systems bestätigen den Erfolg der geleisteten Arbeit. Ich bin zuversichtlich, dass der Feasibility Check einen signifikanten Beitrag zur Effizienzsteigerung in der Produktion leisten wird und freue mich darauf, die weiteren Entwicklungen und das Feedback der Nutzer zu verfolgen.

Abschließend möchte ich mich bei meinem Team und meinen Betreuern für die Unterstützung und das Vertrauen bedanken, das sie mir während des gesamten Projekts entgegengebracht haben. Die Erfahrungen, die ich in diesem Projekt sammeln konnte, haben nicht nur meine technischen Fähigkeiten gestärkt, sondern auch mein Interesse an der Weiterentwicklung innovativer Softwarelösungen vertieft. Ich bin zuversichtlich, dass die gewonnenen Erkenntnisse und Fähigkeiten mir in meiner zukünftigen beruflichen Laufbahn von großem Nutzen sein werden.




