\chapter{Fazit}

Das entwickelte Feasibility Check-System stellt eine zuverlässige und praxistaugliche Lösung für die Durchführung technischer Feasibility Checks in der Halbleiterproduktion dar. Es bietet einen echten Mehrwert für die Anwender und unterstützt effizient die Entscheidungsprozesse.

Ein zentraler Erfolg des Projekts ist die erfolgreiche Umsetzung der Anwendung, die sich derzeit in der ''Staging-Phase'' befindet und in Kürze produktiv eingesetzt wird. Dies bestätigt die Praxistauglichkeit und zeigt, dass die Lösung nicht nur theoretisch konzipiert, sondern tatsächlich einsatzbereit ist.

Die enge und produktive Zusammenarbeit im Team war ein entscheidender Faktor für den Projekterfolg. Durch regelmäßige Abstimmungen, offene Kommunikation und den Austausch von Wissen konnten Herausforderungen effizient bewältigt und gemeinsam Lösungen erarbeitet werden. Diese strukturierte Vorgehensweise trug maßgeblich dazu bei, dass das Projekt in hoher Qualität abgeschlossen wurde.

Ein weiterer wichtiger Aspekt der Arbeit war die strukturierte Planung und Dokumentation, die eine klare Nachvollziehbarkeit und Transparenz des Entwicklungsprozesses gewährleisteten. Die Erstellung von klaren Anforderungen, die Entwicklung von robustem Code und die Implementierung von umfangreichen Unittests waren entscheidend, um eine zuverlässige und fehlerfreie Software zu schaffen. Diese Maßnahmen haben dazu beigetragen, dass das System den hohen Qualitätsstandards entspricht.

Zusammenfassend lässt sich festhalten, dass der umgesetzte Feasibility Check alle gestellten Anforderungen erfolgreich erfüllt und darüber hinaus eine solide Grundlage für zukünftige Erweiterungen und Optimierungen bietet. Die positiven Rückmeldungen und die bevorstehende produktive Nutzung bestätigen den Erfolg der geleisteten Arbeit. Das System wird voraussichtlich einen bedeutenden Beitrag zur Effizienzsteigerung in der Produktion leisten und lässt sich durch gezielte Weiterentwicklungen noch stärker an die Bedürfnisse der Qualitätstests in der Halbleiterindustrie anpassen.




