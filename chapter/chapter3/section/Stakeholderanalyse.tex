\section{Stakeholder-Analyse}
Andreas Gauditz hat die Rolle eines LabAdmins im \gls{RPT}-Labor in Regensburg. Er unterstützt die \gls{RPT}-Labor \gls{operator} bei ihrer Arbeit und hilft bei Problemen im Umgang mit \gls{REALIS}. Außerdem entwickelt er die Projekt-Test-Templates in \gls{REALIS}, die die \gls{QM} beim Anlegen von Tests in Projekten unterstützen.
Auch arbeitet er eng mit dem Entwicklerteam von \gls{REALIS} zusammen und beteiligt sich beim Testen von neuen Versionen von der Software.

Andreas Gauditz erhofft sich für den neuen technischen Feasibility Check eine automatisierte Version des bisherigen Feasibility Checks, bei dem die \gls{RPT}-Labor Mitarbeiter keine Arbeit mehr leisten müssen, vorausgesetzt der Feasibility Check ist erfolgreich. Ansonsten möchte er gerne eine "Feasibility Worklist" in dem alle Tests angezeigt werden, die den Feasibility Check durchlaufen oder durchlaufen haben. Hier möchte er ein Interfglse bei dem \gls{operator} beim Fehlschlagen des automatischen Feasibility Checks eingreifen können, und diese Checks wie bisher manuell bearbeiten können.
