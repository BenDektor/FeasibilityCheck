\section{Stakeholder-Analyse}
Im Rahmen der Entwicklung des automatisierten technischen Feasibility Checks stand ich mehrfach wöchentlich im Austausch mit den Stakeholdern. Diese setzten sich aus vier Schlüsselpersonen sowie einer Personengruppe zusammen.

Zu den Einzelpersonen gehörten der Product Owner von \gls{REALIS}, der Application Owner, der Scrum Master sowie ein LabAdmin des \gls{RPT}-Labors in Regensburg. Ergänzt wurde diese Gruppe durch das externe Entwicklungsteam von \gls{REALIS}, dessen Mitglieder in Portugal ansässig sind.

Der \textbf{Product Owner} von \gls{REALIS} vertritt die Interessen und Anforderungen aller Stakeholder, darunter die \gls{RPT}-Labore und \glspl{QM}, um ein gemeinsam abgestimmtes Produktziel zu entwickeln. Seine Hauptaufgabe besteht darin, diese Anforderungen zu priorisieren und in klar formulierte Einträge des Product Backlogs für das Entwicklungsteam zu überführen. Dabei stellt er sicher, dass das Team kontinuierlich an den wichtigsten Aufgaben arbeitet \cite{scrumguide2020}.

Der \textbf{Application Owner} ist für die strategische Ausrichtung und langfristige Verwaltung von \gls{REALIS} verantwortlich. Seine Aufgaben umfassen die Optimierung der Anwendung und die Sicherstellung, dass sie den geschäftlichen Anforderungen gerecht wird. Dies beinhaltet die Gewährleistung von Stabilität und Sicherheit sowie die Planung von Systemaktualisierungen, beispielsweise den Wechsel vom Desktop-Frontend zu einer webbasierten Anwendung.

Der \textbf{Scrum Master} moderiert die agilen Prozesse und achtet darauf, dass diese korrekt eingehalten werden. Er unterstützt das Team dabei, sich kontinuierlich zu verbessern und effizienter zu arbeiten. Zudem fördert er die Zusammenarbeit zwischen den Teammitgliedern und sorgt für die Schaffung eines produktiven Arbeitsumfelds. Hierfür beseitigt er Hindernissen zwischen den Stakeholdern und dem Scrum Team \cite{scrumguide2020}.

Der \textbf{LabAdmin} im \gls{RPT}-Labor in Regensburg unterstützt die \gls{RPT}-Labor-\glspl{operator} bei ihrer Arbeit und hilft bei der Lösung von Problemen im Umgang mit \gls{REALIS}. Darüber hinaus entwickelt er Projekt-Test-Templates in \gls{REALIS}, die \glspl{QM} beim Anlegen von Tests in Projekten unterstützen. Er ist zudem aktiv in das Testen neuer Softwareversionen von \gls{REALIS} eingebunden.


Das \textbf{externe Entwicklungsteam} von \gls{REALIS}, mit Sitz in Portugal, stellt sicher, dass die Coding-Standards eingehalten werden. Es legt besonderen Wert auf eine sichere und gut dokumentierte Programmierung, um die Wartbarkeit und Weiterentwicklung des Codes zu gewährleisten.