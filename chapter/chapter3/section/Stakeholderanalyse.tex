\section{Stakeholder-Analyse}
Andreas Gauditz hat die Rolle eines LabAdmins im \gls{RPT}-Labor in Regensburg. Er unterstützt die \gls{RPT}-Labor \gls{operator} bei ihrer Arbeit und hilft bei Problemen im Umgang mit \gls{REALIS}. Außerdem entwickelt er die Projekt-Test-Templates in \gls{REALIS}, die die \gls{QM} beim Anlegen von Tests in Projekten unterstützen.
Auch arbeitet er eng mit dem Entwicklerteam von \gls{REALIS} zusammen und beteiligt sich beim Testen von neuen Versionen von der Software.

Andreas Gauditz erhofft sich für den neuen technischen Feasibility Check eine automatisierte Version des bisherigen Feasibility Checks, bei dem die \gls{RPT}-Labor Mitarbeiter keine Arbeit mehr leisten müssen, sondern dass System über die Kapaziäten eine Labors Bescheid weiß und so automatisiert ablaufen kann. Ansonsten möchte er gerne eine "Feasibility Worklist" Frontend in dem alle Tests angezeigt werden, die den Feasibility Check durchlaufen oder durchlaufen haben. Hier möchte er ein Interface bei dem \gls{operator} beim Fehlschlagen des automatischen Feasibility Checks eingreifen können, und diese Checks wie bisher manuell bearbeiten können.

Florian Saller (Scrum Master) ist Moderator dafür da das agile Prozesse eingehlaaten werden. Dafür da dass sich das Team verbessert um effizienter zu werden . ist mein direkter Betreuer für die Bachelorarbeit. Er ist IT Application Owner von \gls{REALIS} und arbeitet deswegen seit Jahrzehnten an der (Weiter-)Entwicklung von \gls{REALIS}. Er kennt sich besonders gut mit der Datenbankstruktur des Systems aus, und ist daran interessiert, dass das System in seiner Struktur vereinfacht wird.


Fabian Vilsmeier (Applikation Owner) ist mein zweiter Betreuer, unter dem ich in meiner Werkstudenten Zeit bei Infineon zusammengearbeitet habe. Er hat angefangen mit seiner Masterarbeit eine MobileApp für \gls{REALIS} zu entwickeln. Diese stellt den Anfang des \texttt{REALIS-Web-Operator} Systems dar, dass spezifisch für die \gls{operator} in \gls{RPT}-Laboren entwickelt wird. Bei diesem habe ich während meiner Werkstudententätigkeit bei Infineon mitentwickelt, anfangs nur als MobileApp mit dem Framework Xamarin später sollte diese ebenfalls als WebApp verfügbar sein, weswegen dann auf Angular umgestellt wurde.
Aktuell ist er im IT Application Team von \gls{REALIS} und soll in Zukunft immer mehr Aufgaben von Florian Saller im Bezug auf \gls{REALIS} übernehmen.


Thomas Gombotz ist Product Owner und Buisness Applikation Owner; bringt Feature Request, entscheidet üer priosiersuierng was gemacht werden soll , er redet mit stakeholder um anforderungen einzuholen. Er versucht die Interessen und Anforderungen von allen \gls{RPT}-Laboren und auch \glspl{QM} auszuarbeiten und zu priorisieren. Dabei stellt er zusammengefassten Anforderungen an das Entwicklungsteam von \gls{REALIS}.

Die eigentlichen Programmierer von \gls{REALIS} bestehen aus einem externen Team von Entwicklern, die ihren Sitz in Portugal haben. Diese sind daran interessiert, dass ich bei der Programmierung ihre Coding Standarts einhalte und außerdem den Code sicher programmiere und ausführlich kommentiere, damit sie nach meiner Arbeit keine Probleme damit haben, diesen falls nötig zu ändern oder weiterzuentwickeln. 