\section{Stakeholder-Analyse}
Im Rahmen der Entwicklung des automatisierten technischen Feasibility Checks erfolgte mehrfach wöchentlich ein Austausch mit den Stakeholdern. Diese setzen sich aus vier Schlüsselpersonen sowie einem Entwicklungsteam zusammen.

Zu den Schlüsselpersonen gehören der Product Owner von \gls{REALIS}, der Application Owner, der Scrum Master sowie ein LabAdmin des \gls{RPT}-Labors in Regensburg. Ergänzt wird diese Gruppe durch das externe Entwicklungsteam von \gls{REALIS}, dessen Mitglieder in Portugal ansässig sind.

Der \textbf{Product Owner} von \gls{REALIS} fungiert als Interessenvertreter der Stakeholder, zu denen die \gls{RPT}-Labore und \glspl{QM} zählen, mit dem Bestreben, ein konsentiertes Produktziel zu entwickeln. Seine Hauptaufgabe besteht darin, diese Anforderungen zu priorisieren und in klar formulierte Einträge des Product Backlogs für das Entwicklungsteam zu überführen. Dabei stellt er sicher, dass das Team kontinuierlich an den wichtigsten Aufgaben arbeitet \cite{scrumguide2020}.

Der \textbf{Application Owner} trägt die Verantwortung für die strategische Ausrichtung und die langfristige Verwaltung von \gls{REALIS}. In seiner Rolle als Applikations-Architekt sorgt er für die Einhaltung der Architektur-Richtlinien. Zu seinen Aufgaben gehören die kontinuierliche Optimierung der Anwendung, die Sicherstellung ihrer Stabilität und Sicherheit sowie die Planung von Systemaktualisierungen, wie etwa der Wechsel vom Desktop-Frontend zu einer webbasierten Anwendung, um den geschäftlichen Anforderungen gerecht zu werden.

Der \textbf{Scrum Master} moderiert die agilen Prozesse und achtet darauf, dass diese korrekt eingehalten werden. Er unterstützt das Team dabei, sich kontinuierlich zu verbessern und effizienter zu arbeiten. Zudem fördert er die Zusammenarbeit zwischen den Teammitgliedern und schafft ein produktives Arbeitsumfeld. Hierfür beseitigt er Hindernisse zwischen den Stakeholdern und dem Scrum Team \cite{scrumguide2020}.

Der \textbf{LabAdmin} im \gls{RPT}-Labor in Regensburg assistiert den \gls{RPT}-Labor-\glspl{operator} bei ihrer Tätigkeit und hilft bei der Lösung von Problemen im Umgang mit \gls{REALIS}. Darüber hinaus entwickelt er Projekt-Test-Templates in \gls{REALIS}, die \glspl{QM} beim Anlegen von Tests in Projekten unterstützen. Er ist zudem aktiv in das Testen neuer Software-Versionen von \gls{REALIS} eingebunden.


Das \textbf{externe Entwicklungsteam} von \gls{REALIS} mit Sitz in Portugal stellt sicher, dass die Coding-Standards eingehalten werden. Es legt besonderen Wert auf eine stabile und gut dokumentierte Programmierung, um die Wartbarkeit und Weiterentwicklung des Codes zu gewährleisten.