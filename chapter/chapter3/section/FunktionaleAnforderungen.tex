\section{Funktionale Anforderungen}

\setlength{\leftskip}{1em} 
\textbf{Automatisierung des technischen Feasibility Checks}  \\
Der Feasibility Check soll vollständig automatisiert ablaufen, sodass Benutzer keine technischen Vorkenntnisse benötigen. Die Ausführung wird ausschließlich durch den Benutzer initiiert oder angefordert.

\textbf{Nachvollziehbarkeit}  \\
Um das Ergebnis des automatisierten Feasibility Checks transparent zu gestalten, müssen alle zugrunde liegenden logischen Entscheidungen in der Datenbank gespeichert werden. Diese Informationen sollen für den Benutzer in verständlicher Form dargestellt werden, insbesondere wenn die Logik auf unerwartete Probleme stößt.

\textbf{Modularer Aufbau}  \\
Der Feasibility Check besteht aus mehreren separaten ''Checks''. Diese Module, abhängig von den Parametern (siehe Kapitel~\ref{Subsec:ParameterdestechnischenFeasibilityChecks}), sollen von Systemadministratoren individuell aktiviert oder deaktiviert werden können. Ist ein Modul deaktiviert, soll der Benutzer weiterhin die Möglichkeit haben, den Check manuell durchzuführen.

\textbf{Flexibilität und Erweiterbarkeit}  \\
Das System soll so konzipiert sein, dass neue Module bzw. ''Checks'' flexibel hinzugefügt werden können, um zukünftige Anforderungen ohne größeren Aufwand zu erfüllen.

\textbf{Manuelle Durchführung weiterhin Ermöglichen}  \\
Falls der automatisierte Feasibility Check zu einem negativen Ergebnis kommt, auch aufgrund von Problemen in der bestehenden Logik, soll der Benutzer die Möglichkeit haben, den Check manuell durchzuführen.

\textbf{Integration in bestehende Architektur}  \\
Die Systemarchitektur des Feasibility Checks muss vollständig kompatibel mit der bestehenden Architektur von \gls{REALIS} sein, um eine nahtlose Integration zu gewährleisten.

\setlength{\leftskip}{0em} 
