\section{Funktionale Anforderungen} \label{Sec:funktionale-anforderungen}

\setlength{\leftskip}{1em} 
\textbf{Automatisierung des technischen Feasibility Checks}  \\
Der technische Feasibility Check soll vollständig automatisiert erfolgen, sodass keine technischen Vorkenntnisse seitens der Benutzer erforderlich sind. Die Ausführung wird ausschließlich durch den Benutzer initiiert oder angefordert. Durch die Automatisierung sollen menschliche Fehler reduziert, und die Genauigkeit der Entscheidungen deutlich gesteigert werden.

\textbf{Nachvollziehbarkeit}  \\
Um die Ergebnisse des automatisierten Feasibility Checks transparent und nachvollziehbar zu machen, müssen alle zugrunde liegenden logischen Entscheidungen in verständlicher Form für den Benutzer aufbereitet werden. Dies ist besonders wichtig, wenn die Logik auf unerwartete Probleme stößt. Darüber hinaus ermöglicht die Automatisierung eine fundierte Begründung der Entscheidungen, die über die bisherige einfache „Ja/Nein“-Auswahl hinausgeht.

\textbf{Modularer Aufbau}  \\
Der Feasibility Check besteht aus mehreren separaten ''Checks'' (wie Condition und Equipment Check). Diese Module, abhängig von den Parametern (siehe Kapitel~\ref{Subsec:ParameterdestechnischenFeasibilityChecks}), sollen von Systemadministratoren individuell aktiviert oder deaktiviert werden können. Ist ein Modul deaktiviert, soll der Benutzer weiterhin die Möglichkeit haben, den Check manuell durchzuführen.

\textbf{Flexibilität und Erweiterbarkeit}  \\
Das System soll so konzipiert sein, dass neue Module bzw. ''Checks'' flexibel hinzugefügt werden können, um zukünftige Anforderungen ohne größeren Aufwand zu erfüllen.

\textbf{Manuelle Durchführung weiterhin Ermöglichen}  \\
Falls der automatisierte Feasibility Check zu einem negativen Ergebnis kommt, auch aufgrund von Problemen in der bestehenden Logik, soll der Benutzer die Möglichkeit haben, den Check manuell durchzuführen.

\textbf{Integration in bestehende Architektur}  \\
Die Systemarchitektur des Feasibility Checks muss vollständig kompatibel mit der bestehenden Architektur von \gls{REALIS} sein, um eine nahtlose Integration zu gewährleisten.

\setlength{\leftskip}{0em} 
