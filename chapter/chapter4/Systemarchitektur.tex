\chapter{Systemdesign}

Da der Feasibility Check als Funktionalität in die bestehende Software-Applikation \gls{REALIS} integriert werden soll, ist es notwendig, die bereits verwendeten Technologien und Programmiersprachen sowie die Architektur zu übernehmen, um eine nahtlose Integration zu gewährleisten.

Eine Evaluierung alternativer Technologien, die möglicherweise besser für die Applikation geeignet wären, wurde daher nicht durchgeführt. Stattdessen orientiert sich die Umsetzung strikt am bestehenden System.

Für das Datenbankdesign kommt eine Oracle-Datenbank mit SQL zum Einsatz. Die Backend-Logik wird in der objektorientierten Programmiersprache C\# implementiert. Für das Frontend wird das Webapplikationsframework Angular verwendet, das auf TypeScript basiert. Angular ermöglicht eine parallele Entwicklung sowohl für Webanwendungen als auch für native Mobilapplikationen.

\section{Architekturüberblick und Integration}

Die Abbildung \ref{fig:feasibility-check-komponentendiagramm} veranschaulicht die erweiterte REALIS-Architektur zur Realisierung des Feasibility Checks. Basierend auf dem bestehenden REALIS-Komponenten\-diagramm (siehe Abbildung 2.3) werden neue Komponenten hinzugefügt, um die Feasibility-Check-Funktionalität zu integrieren. Im Diagramm sind dabei aus Gründen der Übersichtlichkeit nur die wichtigsten Erweiterungen dargestellt.

\begin{figure}[!htbp]
    \centering
    \includegraphics[width=1\textwidth]{bilder/REALIS-Komponentendiagramm-mit-Erweiterungen2.png}
    \caption{Feasibility Check Architekturdesign-Erweiterung von \gls{REALIS}}
    \label{fig:feasibility-check-komponentendiagramm}
\end{figure}



In der \textbf{Datenbank} werden neue Tabellen hinzugefügt, die sowohl zur Konfiguration als auch zur Speicherung der Feasibility Check Ergebnisse dienen. Hierzu gehören die Tabellen \texttt{ctlg\_feasibility\_config}, \texttt{feasibility\_check\_item\_result} und \texttt{check\_item\_\-result\_\-list}.

Im \textbf{Backend} wird eine HTTP-GET-Methode (in Abb. \ref{fig:feasibility-check-komponentendiagramm} und \ref{fig:sequence-diagram} mit ''Helfer-Methode'' benannt) implementiert. Diese Methode ruft den Kern-Algorithmus, die Funktion \texttt{FeasibilityCheck()} auf, die die eigentliche Datenverarbeitung übernimmt, die Ergebnisse der Feasibility Check Überprüfung in der Datenbank abspeichert und diese über die ''Helfer-Methode'' an das Frontend liefert.

Im \textbf{Frontend} wird im \textit{REALIS-Web Subsystem} eine neue Modal-Page ''Submit for Feasibility Check'' integriert, die es Benutzern ermöglicht, den Feasibility Check zu initiieren. Nach der Überprüfung werden die Ergebnisse übersichtlich dargestellt und zusätzliche Details können bei Bedarf abgerufen werden. 

Durch diese Erweiterungen wird die Feasibility-Check-Funktion nahtlos in die bestehende REALIS-Architektur eingebettet und ermöglicht eine effiziente Interaktion mit den relevanten Benutzern und Systemkomponenten.

Die wichtigsten Interaktionen zwischen Benutzer (\gls{QM}), Frontend, Backend und der Datenbank sind im Sequenzdiagramm in Abbildung \ref{fig:sequence-diagram} dargestellt. Die einzelnen Komponenten werden in den folgenden Kapiteln detailliert beschrieben. 

\begin{figure}[!htbp]
    \centering
    \includegraphics[width=1\textwidth]{bilder/Feasibility-Sequenz-Diagramm.png}
    \caption{Sequenzdiagramm Feasibility Check}
    \label{fig:sequence-diagram}
\end{figure}




\section{Datenbankdesign}
Um die Erweiterung der Datenbank für den Feasibility Check zu erklären, wird zunächst auf die grundlegende Struktur der REALIS-Datenbank eingegangen. In Abbildung \ref{fig:realis-datenbankdesign} ist ein Ausschnitt der wichtigsten Tabellen und deren Beziehungen dargestellt, einschließlich der vorgenommenen Erweiterungen. Die nachfolgenden Erläuterungen nennen jeweils den zugehörigen Tabellennamen aus der Abbildung in Klammern.

\subsection{REALIS-Datenbank}

Wie bereits in Kapitel \ref{Subsec:project-lifecycle} beschrieben, besteht ein \textbf{REALIS-Projekt} (\texttt{relproject}), in dem neue Produkte qualifiziert werden, aus mehreren \textbf{Tests} (\texttt{tests}). Jeder Test besitzt einen eindeutigen \textbf{State} (\texttt{ctlg\_state\_type}), der zu Beginn auf ''NEW'' gesetzt wird und nach Abschluss den Status ''ARCHIVE'' erhält (vgl. Abbildung \ref{fig:realis-project-lifecycle}, rechte Spalte). Zudem wird jeder Test einer \textbf{Kategorie} bzw. einem \textbf{Test-Typ} (\texttt{ctlg\_test\-\_sub\_type}) zugeordnet.

Ein Test besteht aus mehreren aufeinanderfolgenden \textbf{Operationen} (\texttt{operation}), die nacheinander im \gls{RPT}-Labor abgearbeitet werden. Die Tabellen, die eng mit der Operation verknüpft sind, sind in Abbildung \ref{fig:realis-datenbankdesign} im grün umrandeten Bereich dargestellt. Jede Operation kann dabei einen oder mehrere \textbf{Parameter} (\texttt{op\_data}) enthalten, die beispielsweise die Umgebungsbedingungen definieren. Die konkreten Werte dieser Parameter sind in dem Eintrag \texttt{OPD\_PLAN} hinterlegt. Jeder Parameter wird zudem einer \textbf{Kategorie} (\texttt{ctlg\_op\_\-data\_type}) zugeordnet.  


\begin{figure}[!htbp]
    \centering
    \makebox[\textwidth]{\includegraphics[width=0.98\paperwidth]{bilder/REALIS-Datenbankmodell2.png}}
    \caption{REALIS Datenbankdesign}
    \label{fig:realis-datenbankdesign}
\end{figure}

Die Tabelle \texttt{op\_name\_name\_type} bestimmt welche Arten von Operationen es gibt, und legt indirekt über die Tabelle \texttt{ctlg\_op\_loc\_type} fest, welche Operations-Typen auf welchen Standort verfügbar sind. Darüber hinaus definiert die 
\texttt{op\_data\_param\_type} Tabelle, welcher Parameter bei welcher Operation auf welchem Standort für welchen Produkttypen (\texttt{ctlg\_product\_type}) verfügbar ist. Hierbei sind der Standort und der Produkttyp aber nur optionale Einträge, um generische, standortübergreifende und/oder produktübergreifende Einträge zu realisieren.

Jeder (Stress-)Parameter einer Operation, die einen konkreten Wert definiert (\texttt{OPD\_PLAN}) muss von einer Maschine \texttt{ctlg\_machine} umgesetzt werden (können). Die eng mit der Maschine verknüpften Tabellen, befinden sich in der Abbildung \ref{fig:realis-datenbankdesign} im blau umrandeten Bereich. Eine Maschine ist verknüpft mit einem Machinen-Test-Typen (\texttt{machine\_test\_sub\_type}), der die Machine einem Test-Typen zuweist (\texttt{ctlg\_test\_sub\_type}). Außerdem besitzen Maschinen verschiedene ''Factsheets'' (\texttt{machine\_factsheet}) mit einem Parameter (\texttt{ctlg\_factsheet\_parameter}) und einem Wert (\texttt{MFS\_VALUE}). Dieser Wert korrespondiert mit der PlanValue (\texttt{OPD\_PLAN}) des Operations-Parameters. Hierbei gibt es aber (meist) für jeden Operations-Parameter-Typen zwei zugehörige Maschinen-Factsheet-Parameter(-Typen), die angeben, was jeweils das Minimum und das Maximum der Maschine, bei diesem Parameter ist.


\subsection{Datenbankerweiterung mit Feasibility Check}

Für den Feasibility Check muss die Datenbank nun erweitert werden. Hierbei müssen die Parameter und die Logik des Feasibility Checks, die in Kapitel \ref{Subsec:ParameterdestechnischenFeasibilityChecks} beschrieben worden sind, ermöglicht werden. Zusätzlich dazu sollen die in Kapitel \ref{Chap:Anforderungen} besprochenen Anforderungen eingehalten werden.

Für den Feasibility Check soll der Wert des Parameters der (Stress-)Operation, zum einen auf Sinnhaftigkeit und zum anderen auf Durchführbarkeit überprüft werden. Dieser Wert entspricht im Datenbankdesign von \gls{REALIS} dem Eintrag \texttt{OPD\_PLAN} in der Tabelle \texttt{op\_data}, die für den Parameter steht.

Damit dieser Wert auf Sinnhaftigkeit überprüft werden kann, werden in der generischen Tabelle \texttt{opd\_data\_param\_type}, die Operationstyp, Parametertyp, Standort und Produkttyp miteinander verknüpft, zwei neue Einträge hinzugefügt. Diese zwei Einträge beinhalten Werte, die den ''sinnvollen'' Bereich festlegen. Der Eintrag \texttt{OPT\_CD\_MIN} legt dabei die untere Grenze des Bereichs fest, also den minimalen möglichen Wert. Und der Eintrag \texttt{OPT\_CD\_MAX} definiert die obere Grenze, also den maximalen Wert.

Ähnlich wird das ganze für die Überprüfung der Durchführbarkeit angelegt. Hierbei wird die generische Tabelle ebenfalls durch zwei neue Einträge für das Minimum (\texttt{OPT\_EQ\_MIN\_ID}) und das Maximum (\texttt{OPT\_EQ\_MAX\_ID}) erweitert. Diese beinhalten aber nicht direkt zwei Werte, sondern sind Referenzen auf zwei Maschinen-Parametertypen (\texttt{ctlg\_factsheet\_parameter}), wobei einer der beiden dem ''minimalen'' Parameter und der andere dem ''maximalen'' Parameter entpricht. Die tatsächlichen Werte stehen dann in der verknüpften \texttt{machine\_factsheet} Tabelle, in dem Feld \texttt{MFS\_VALUE}. Dies wird so gelöst, da, wie im vorherigen Kapitel schon beschrieben, für jeden Operations-Parameter-Typen jede Maschine, die diesen Parameter-Typen umsetzten kann, zwei korrespondierende FactSheet-Parameter(-Typen) besitzt, in die in der mit der Maschine verknüpften Machine-Factsheet Tabelle über einen Foreign-Key verwiesen wird. Diese zwei Machine-FactSheet(-Typen) beschrieben dabei den minimalen und maximalen ''Typen'', wobei der tatsächlichen minimale und maximale Wert, den die Maschine für diesen Typen umsetzten kann, in dem Machine-Factsheet steht.



\section{Backend-Logik}\label{Sec:Backend-Logik}

Die Logik des Feasibility-Check-Algorithmus besteht aus vier zentralen Methoden, die zusammen die Prüfung der Machbarkeit eines Tests realisieren. Zum einen gibt es eine Helfer-Methode, die den eigentlichen Algorithmus aufruft und sicherstellt, dass der Aufrufer der REST-API eine aussagekräftige Rückmeldung erhält. Die Kernkomponente bildet die \texttt{FeasibilityCheck()}-Methode, welche abhängig von der Art der Überprüfung den Condition oder den Equipment Check aufruft. Beide Prüfungen sind als eigenständige Methoden implementiert. Der gesamte Ablauf wird durch vier Aktivitäts- bzw. Flussdiagramme visualisiert. In diesen Diagrammen werden Schleifen durch farblich hervorgehobene Bereiche dargestellt, die jeweils einen gleichfarbigen Kreis als Einstiegspunkt sowie einen zweiten als Ausstiegspunkt enthalten.

Die Ausführung des Algorithmus wird initiiert, sobald der Benutzer im Frontend einen \textit{Button} in der Weboberfläche betätigt. Dies löst einen asynchronen HTTP-Call über eine REST-API aus, welcher die Helfer-Methode \texttt{CheckFeasibility()} aufruft und die Test-ID des zu überprüfenden Tests als Argument übergibt.

\begin{figure}[!htbp]
    \centering
    \makebox[\textwidth]{\includegraphics[width=0.85\paperwidth]{bilder/flowchart-check-feasibility-http-call-6-2.png}}
    \caption{Flussdiagramm der Helfer-Methode \texttt{CheckFeasibility()}}
    \label{fig:feasibility-http-call-method}
\end{figure}

Der Ablauf dieser Helfer-Methode ist in Abbildung \ref{fig:feasibility-http-call-method} dargestellt. Ihre Hauptaufgabe besteht darin, die eigentliche \texttt{FeasibilityCheck()}-Methode (grünes Rechteck in der Graphik \ref{fig:feasibility-http-call-method}) auszuführen und zusätzlich eine geeignete HTTP-Statusmeldung an das Frontend bzw. den Benutzer zurückzugeben. Zusätzlich wird vor der Verarbeitung die übergebene Test-ID auf Gültigkeit überprüft und der Test-Status auf \texttt{FEASIBILITYREQUESTED} aktualisiert. Falls der FeasibilityCheck für diesen Test bereits zuvor durchgeführt wurde, müssen alte Ergebnisse in der Datenbank noch als ''invalid'' markiert.


Zur besseren Übersicht wird in Abbildung \ref{fig:feasibility-http-call-method} auch der initiale Button-Klick im Frontend als Aktivität (lila markiert) dargestellt. Die eigentliche Logik beginnt jedoch erst nach dem schwarzen Einstiegspunkt, der den Start des Algorithmus kennzeichnet.

\begin{figure}[!htbp]
    \centering
    \makebox[\textwidth]{\includegraphics[width=0.85\paperwidth]{bilder/flowchart-feasibilitycheck-modiefied-for-thesis-6-2.png}}
    \caption{Flussdiagramm des Feasibility Check}
    \label{fig:feasibility-check}
\end{figure}

Die Kernlogik des Feasibility Checks wird durch die Methode \texttt{FeasibilityCheck()} ausgeführt, die im Flussdiagramm in Abbildung \ref{fig:feasibility-check} detailliert dargestellt ist. Diese Methode erhält von der Helfer-Methode die Test-ID und überprüft die \textit{Feasibility} des Tests anhand der relevanten Parameter und Operationen. Um die Logik des Algorithmus zu vereinfachen, wird ein Enum eingeführt, das den Status einzelner \textit{Teil-Checks} kennzeichnet. Dieser besteht aus vier möglichen Stati mit jeweils spezifischer Bedeutung:

\begin{table}[htbp]
    \centering
    \footnotesize
    \renewcommand{\arraystretch}{1.3} % Erhöht den Zeilenabstand
    \begin{tabular}{p{0.25\linewidth} p{0.7\linewidth}}
        \toprule
        \textbf{Status} & \textbf{Beschreibung} \\
        \midrule
        \texttt{CHECK\_OFF} & Der Check ist deaktiviert; eine Überprüfung ist weder erforderlich noch vorgesehen. Dies ist insbesondere der Fall, wenn ein Parameter bzw. eine Operation keine Stressoperation darstellt und somit keine PlanValue definiert werden muss. \\
        \midrule
        \texttt{CHECK\_OK} & Der Check war erfolgreich. \\
        \midrule
        \texttt{CHECK\_NOT\_OK} & Der Check war nicht erfolgreich oder es ist ein Fehler aufgetreten. \\
        \midrule
        \texttt{CHECK\_MANUAL} & Eine automatische Überprüfung ist nicht möglich oder der Check ist noch nicht für die automatisierte Überprüfung zugelassen; der Check muss von einem Mitarbeiter manuell durchgeführt werden. \\
        \bottomrule
    \end{tabular}
    \caption{Stati von Feasibility-Teil-Checks, gespeichert als \texttt{FCIR\_STATE} in Feasibility Result (\texttt{feasibility\_check\_item\_result}))}
    \label{tab:feasibility-states}
\end{table}

Jede zu überprüfende Einheit – sei es auf der Ebene des Parameters, der Operation oder des CheckItems – erhält einen entsprechenden Status. Der Algorithmus arbeitet hierarchisch: Zunächst erfolgt die Bewertung auf der untersten Ebene, also bei den Parametern, sofern diese vorhanden sind. Enthält ein Test Operationen, so werden diese Operationen und deren Parameter einzeln bewertet. Für jeden Parameter, der ein prüfbares Ergebnis liefert, wird ein Feasibility-Ergebnis in der Datenbank gespeichert. Enthält ein Test zwar Operationen, diese jedoch keine Parameter, erfolgt die Speicherung ausschließlich auf der Ebene der Operation. Besitzt ein Test gar keine Operationen (und somit auch keine Parameter), wird lediglich ein Gesamtergebnis für den Test abgelegt.

Obwohl die Ergebnisse der untergeordneten Ebenen zu einem Gesamtstatus aggregiert werden – beispielsweise durch die Zusammenfassung der Parameterwerte zu einem Operationsergebnis (siehe orange markierte Aktivitäten in Abb. \ref{fig:feasibility-check}) und der Operationsergebnisse zu einem CheckItem-Ergebnis (sei es ein \gls{ConditionCheck} oder \gls{EquipmentCheck}, dargestellt in den grünen Aktionen) – wird in der Datenbank ausschließlich das Ergebnis der jeweils untersten vorhandenen Ebene persistent abgelegt. Durch diese Maßnahme in Kombination mit der bewussten Nicht-Speicherung von Ergebnissen, die lediglich den Status \texttt{CHECK\_OFF} aufweisen, wird eine unnötige Datenbankaufblähung vermieden und die Systembelastung reduziert, da diese Ergebnisse keine zusätzlichen, benutzerrelevanten Informationen liefern. Zudem wird der Persistierungsvorgang in den Flussdiagrammen nicht explizit dargestellt, da die Speicherung auf unterschiedlichen Ebenen erfolgen und eine detaillierte Visualisierung die Diagramme erheblich komplizierter machen würde.

Basierend auf den Ergebnissen aller CheckItems wird schließlich der finale Test-Status bestimmt. Dieser kann entweder den Wert \textbf{''APPLY''} (''FEASIBLE''; erfolgreich) oder \textbf{''MANUALFEASIBILITY''} (manuelle Prüfung erforderlich) (blau markierte Aktivitäten) annehmen.

Um es zu ermöglichen, dass bestimmte \textit{CheckItems} für einen Test in der Konfiguration vollständig deaktiviert werden können – wobei dies im System als erfolgreiche Überprüfung gewertet wird – kann das entsprechende Flag in der Feasibility-Konfiguration entweder leer (\textit{null}) gelassen werden oder es wird für das betreffende CheckItem gar kein Eintrag in der Konfigurations-Tabelle erstellt. \todo{Beispiel}

\textbf{Konkreter Ablauf des Feasibility Check Algorithmus} \\
Zunächst werden die für den Test hinterlegten Konfigurationen für jedes \texttt{CheckItem} aus der Datenbank abgerufen. Falls kein Eintrag existiert, wird der Check automatisch als erfolgreich gewertet. Ist das zugehörige Flag auf 'N' gesetzt, muss die Überprüfung manuell erfolgen, und der Check ist an dieser Stelle beendet. Nur wenn das Flag auf 'Y' gesetzt ist, wird das entsprechende \textit{CheckItem} (Condition oder \gls{EquipmentCheck}) tatsächlich geprüft.

Anschließend wird überprüft, ob der Test mindestens eine Operation enthält. Falls ja, wird jede dieser Operationen durchlaufen, um festzustellen, ob sie einen oder mehrere Parameter besitzt. Für jeden Parameter wird dann – abhängig vom \textit{CheckItem} – entweder der \gls{ConditionCheck} oder der \gls{EquipmentCheck} durchgeführt.

\subsection{Condition Check}



Der Ablauf des \gls{ConditionCheck} ist in Abbildung \ref{fig:condition-check} dargestellt. Zunächst wird die relevante Verknüpfungstabelle \texttt{op\_data\_param\_type} aus der Datenbank abgerufen. Diese Tabelle definiert den zulässigen Wertebereich eines Parameters durch einen minimalen und maximalen Wert. Falls weder ein minimaler noch ein maximaler Wert definiert ist, wird der \textit{Parameter-Teil-Check} auf \texttt{CHECK\_OFF} gesetzt, was bedeutet, dass keine Überprüfung erforderlich ist und der Parameter automatisch als gültig betrachtet wird. Sind sowohl ein minimaler als auch ein maximaler Wert hinterlegt, muss zusätzlich ein geplanter Wert (\textit{PlanValue}) für den Parameter vorhanden sein. Liegt dieser innerhalb des zulässigen Bereichs, gilt die Sinnhaftigkeitsprüfung als erfolgreich, und der nächste Parameter der Operation wird verarbeitet. 

In Sonderfällen, in denen lediglich ein Mindest- oder Höchstwert vorliegt, genügt es, wenn der PlanValue die entsprechende Bedingung (über dem Mindestwert oder unter dem Höchstwert) erfüllt – diese Fälle sind im Flussdiagramm zur besseren Übersichtlichkeit nicht dargestellt.

\begin{figure}[!htb]
    \centering
    \makebox[\textwidth]{\includegraphics[width=0.80\paperwidth]{bilder/flowchart-condition-check-6-2.png}}
    \caption{Flussdiagramm des \gls{ConditionCheck}s}
    \label{fig:condition-check}
\end{figure}

\subsection{Equipment Check}

\begin{figure}[!htbp]
    \centering
    \makebox[\textwidth]{\includegraphics[width=0.85\paperwidth]{bilder/flowchart-equipment-check-6-2.png}}
    \caption{Flussdiagramm des \gls{EquipmentCheck}s}
    \label{fig:equipment-check}
\end{figure}

Der \gls{EquipmentCheck} ist in Abbildung \ref{fig:equipment-check} dargestellt. Der Prozess beginnt ebenfalls mit dem Abruf von Daten aus der passenden \texttt{op\_data\_param\_type} Tabelle. Falls keine Referenzen für den minimalen und maximalen Maschinenparameter (\texttt{ctlg\_factsheet\_parameter}) hinterlegt sind, ist die Überprüfung für diese Parameter, nicht erforderlich und ist somit beendet. Andernfalls muss sichergestellt werden, dass ein \textit{PlanValue} gesetzt ist. Erst danach startet der eigentliche Durchführbarkeits-Check.

Als erstes werden aus der Datenbank alle Maschinen abgerufen, die als ''valide'' gelten, also weder defekt noch in Wartung sind, und zudem den gleichen Testsubtypen aufweisen wie der zu prüfende Test. Die Zuordnung der Maschinen zu einem spezifischen Testsubtyp erfolgt dabei indirekt über die Verknüpfungstabelle \texttt{machine\_test\_sub\_type}, welche die Maschinen mit dem jeweiligen \texttt{ctlg\_test\_sub\_type} verknüpft.

Anschließend werden die verbleibenden Maschinen einzeln durchlaufen (siehe lila markierter Bereich in Abbildung \ref{fig:equipment-check}). Zuerst wird analysiert, ob die Maschine im geplanten Zeitraum der Operation zur Verfügung steht. Danach erfolgt die Überprüfung, ob die Maschine den Parameter grundsätzlich umsetzen kann. Dazu wird geprüft, ob die Maschine über die gleichen \texttt{ctlg\_factsheet\_parameter} verfügt, die in der Verknüpfungstabelle definiert sind. Falls dies der Fall ist, werden die minimalen und maximalen Werte über den zugehörigen Eintrag im Maschinen-Factsheet (\texttt{MFS\_VALUE}) abgerufen. Anschließend wird kontrolliert, ob der geplante Wert (\textit{PlanValue}) des Parameters innerhalb dieser Grenzen liegt. Ist dies der Fall, gilt die Maschine als geeignet, den geplanten Wert umzusetzen, und der Equipment-Check wird als erfolgreich betrachtet.

Alle Maschinen, die den Parameter mit dem geplanten Wert umsetzten können, werden in einer Liste als \texttt{check\_item\_result\_list} erfasst und mit dem zugehörigen \texttt{feasibility\_check\_item\_result} später in der Datenbank abgespeichert. 


\subsection{Benutzer Feedback}

Damit der Endnutzer das Ergebnis des Feasibility Checks nachvollziehen kann, wird jedem Feasibility-Resultat in der Datenbank eine spezifische, nutzerfreundliche Erklärung zugeordnet (\texttt{FCIR\_DESCRIPTION}). Diese Rückmeldungen sind nach den unterschiedlichen Überprüfungsebenen strukturiert, sodass der Benutzer präzise erkennen kann, an welcher Stelle Handlungsbedarf besteht. In folgenden Tabellen \ref{tab:feedback-test}, \ref{tab:feedback-operation}, \ref{tab:feedback-condition} und \ref{tab:feedback-equipment} werden die wichtigsten Rückmeldungen dargestellt.

Feasibility-Results mit dem Status \texttt{CHECK\_OFF} werden im Regelfall nicht gespeichert, da sie keine zusätzlichen, für den Benutzer relevanten Informationen liefern. Die Abspeicherung dieser Ergebnisse kann jedoch zu Debugging-Zwecken temporär aktiviert werden, um eine detailliertere Analyse der Prüfschritte zu ermöglichen.


\begin{table}[htb]
    \centering
    \footnotesize
    \renewcommand{\arraystretch}{1.1} % Erhöht den Zeilenabstand
    \setlength{\arrayrulewidth}{0.1pt} % Dünnere Linien
    \begin{tabular}{p{0.8\linewidth} p{0.15\linewidth}}
        \textbf{Meldung} & \textbf{Status} \\
        \midrule
        Feasibility Configuration not found & \texttt{CHECK\_OFF} \\
        \midrule
        Check Item is disabled for automatic feasibility check, see \texttt{ctlg\_feasibillity\_config} & \texttt{CHECK\_MANUAL} \\
        \midrule
        No Operations found for this test & \texttt{CHECK\_OFF} \\
        \bottomrule
    \end{tabular}
    \caption{Feedback auf Test-Ebene}
    \label{tab:feedback-test}
\end{table}


\begin{table}[htb]
    \centering
    \footnotesize
    \renewcommand{\arraystretch}{1.1}
    \begin{tabular}{p{0.8\linewidth} p{0.15\linewidth}}
        \textbf{Meldung} & \textbf{Status} \\
        \midrule
        Operation with no Parameters & \texttt{CHECK\_OFF} \\
        \bottomrule
    \end{tabular}
    \caption{Feedback auf Operations-Ebene}
    \label{tab:feedback-operation}
\end{table}


\begin{table}[htb]
    \centering
    \footnotesize
    \renewcommand{\arraystretch}{1.1}
    \begin{tabular}{p{0.8\linewidth} p{0.15\linewidth}}
        \textbf{Meldung} & \textbf{Status} \\
        \midrule
        No \texttt{op\_data\_param\_type} was found for this parameter & \texttt{CHECK\_NOT\_OK} \\
        \midrule
        No Conditions defined & \texttt{CHECK\_OFF} \\
        \midrule
        Error: PlanValue is not set & \texttt{CHECK\_NOT\_OK} \\
        \midrule
        Check successful: PlanValue=\{\textit{planValue}\}, ConditionMin=\{\textit{ConditionMin}\}, ConditionMax=\{\textit{ConditionMax}\} & \texttt{CHECK\_OK} \\
        \midrule
        Check failed: PlanValue=\{\textit{planValue}\}, ConditionMin=\{\textit{ConditionMin}\}, ConditionMax=\{\textit{ConditionMax}\} & \texttt{CHECK\_NOT\_OK} \\
        \bottomrule
    \end{tabular}
    \caption{Feedback auf Parameter-Ebene -- Condition Check}
    \label{tab:feedback-condition}
\end{table}


\begin{table}[htb]
    \centering
    \footnotesize
    \renewcommand{\arraystretch}{1.1}
    \begin{tabular}{p{0.8\linewidth} p{0.15\linewidth}}
        \textbf{Meldung} & \textbf{Status} \\
        \midrule
        No \texttt{op\_data\_param\_type} was found for this parameter & \texttt{CHECK\_NOT\_OK} \\
        \midrule
        No Equipment Parameters defined & \texttt{CHECK\_OFF} \\
        \midrule
        Error: PlanValue is not set & \texttt{CHECK\_NOT\_OK} \\
        \midrule
        Check failed: Found no valid machines with Testsubtype of Test & \texttt{CHECK\_NOT\_OK} \\
        \midrule
        Check successful: Found \{\textit{numberOfMachines}\} Machines for: PlanValue=\{\textit{planValue}\}, EquipmentParameterMin=\{\textit{ParameterMin}\}, EquipmentParameterMax=\{\textit{ParameterMax}\} & \texttt{CHECK\_OK} \\
        \midrule
        Check failed: Found no Machines: ... & \texttt{CHECK\_NOT\_OK} \\
        \quad Reason: Found no valid Machine within time period ... & \\
        \quad Reason: Found no valid Machine with needed Parameters ... & \\
        \quad Reason: Current PlanValue=\{\textit{planValue}\} too high: Please set PlanValue lower & \\
        \quad Reason: Current PlanValue=\{\textit{planValue}\} too low: Please set PlanValue higher & \\
        \bottomrule
    \end{tabular}
    \caption{Feedback auf Parameter-Ebene -- Equipment Check}
    \label{tab:feedback-equipment}
\end{table}



Die dargestellten Rückmeldungen ermöglichen es dem Benutzer, gezielt auf erkannte Probleme zu reagieren. So kann er beispielsweise den PlanValue anpassen, wenn in Tabelle~\ref{tab:feedback-equipment} darauf hingewiesen wird, dass der aktuelle Wert zu hoch oder zu niedrig ist. Diese detaillierte Information unterstützt den Benutzer bei der Identifikation und Behebung von Konfigurationsproblemen.




\section{Frontend-Design}
Das Frontend für den Feasibility Check basiert auf der bereits implementierten Weboberfläche des \gls{REALIS}-Systems. Die Funktionalität des Feasibility Checks wird in die bestehende Webanwendung für den \glsentrylong{QM} integriert, sodass das Projekt nicht mehr an das \gls{RPT}-Labor übergeben werden muss, um den Check durchzuführen. Stattdessen ermöglicht das neue, automatisierte System dem \gls{QM}, die Machbarkeit der angelegten Tests eigenständig und ohne Laborunterstützung zu überprüfen.

Zunächst legt der \glsentrylong{QM} ein neues \gls{REALIS}-Projekt für das zu qualifizierende Produkt an. Dieses Projekt kann mit mehreren unterschiedlichen Tests versehen werden, wobei der Nutzer durch vorgefertigte Templates unterstützt wird, die häufig verwendete Testkonfigurationen und zugehörige Daten enthalten.

Nachdem das Projekt vollständig angelegt wurde, gelangt der Benutzer auf die Weboberfläche, wie in Abbildung \ref{fig:whole-page} dargestellt.

\begin{figure}[!htbp] 
    \centering 
    \makebox[\textwidth]{\includegraphics[width=0.85\paperwidth]{bilder/frontend/whole-page.png}} 
    \caption{Weboberfläche für den Quality Manager (QM) – Einstiegspunkt für den Feasibility Check} 
    \label{fig:whole-page} 
\end{figure}

Auf dieser Oberfläche kann der Feasibility Check für die definierten (Stress-)Tests durchgeführt werden. Auf der rechten Seite der Website befindet sich ein grün umrandeter Button mit der Beschriftung "Submit for Feasibility Check". Wird dieser Button betätigt, öffnet sich ein Popup-Fenster (Modal-Page), in dem dem Nutzer eine Übersicht der angelegten Tests in tabellarischer Form präsentiert wird. Diese Tabelle enthält die wichtigsten Informationen, wie den Teststatus und die eindeutige Test-ID (siehe Abbildung \ref{fig:submit-page}).

\subsection{Modal-Page für die Initiierung des Feasibility Checks}

In der Modal-Page können mittels CheckBoxen die einzelnen Tests ausgewählt werden, die den Feasibility Check durchlaufen sollen. Alternativ besteht die Möglichkeit, alle Tests gleichzeitig auszuwählen, indem die oberste CheckBox in der Spaltenüberschriftenleiste aktiviert wird (siehe Mauszeigerposition in Abbildung \ref{fig:submit-page}).

\begin{figure}[!htbp] 
    \centering 
    \makebox[\textwidth]{\includegraphics[width=0.85\paperwidth]{bilder/frontend/feasibility-modal-page.png}} 
    \caption{Modal-Page ''Submit for Feasibility Check'' – Modal-Page zur Initiierung des Feasibility Checks} 
    \label{fig:submit-page} 
\end{figure}

Nachdem der Benutzer die gewünschten Tests selektiert hat, klickt er auf den ''Submit for Feasibility Check''-Button, der oben links im Popup, neben dem ''Exit''-Button, positioniert ist. Mit diesem Klick wird für jeden ausgewählten Test ein asynchroner HTTP-Call an die Backend-Logik initiiert. Über die REST-API wird dabei die Helfer-Methode \texttt{CheckFeasibility()} aufgerufen, wobei die jeweilige Test-ID als Parameter übermittelt wird.

Die asynchronen HTTP-Calls ermöglichen es dem Benutzer, weiterhin mit der Oberfläche zu interagieren, während die Überprüfungen im Hintergrund durchgeführt werden. Dadurch kann der Benutzer das Popup schließen oder andere Aktionen ausführen, ohne den laufenden Feasibility Check zu unterbrechen. Die rotierenden Ladekreise in der "Feasibility"-Spalte rechts veranschaulichen den Ladevorgang für den Feasibility Check jedes Tests (siehe Abbildung \ref{fig:loading-with-results}). Bis ein Resultat eintrifft vergehen circa 2 bis 15 Sekunden, je nach Feasibility-Konfiguration und Komplexität des Tests.

\begin{figure}[!htbp] 
    \centering 
    \makebox[\textwidth]{\includegraphics[width=0.85\paperwidth]{bilder/frontend/loading-with-results.png}} 
    \caption{Modal-Page ''Submit for Feasibility Check'' – Ladevorgang und erste Ergebnisse} 
    \label{fig:loading-with-results} 
\end{figure}

Sobald die ersten Überprüfungen abgeschlossen sind, verschwindet der Ladekreis. Anschließend werden drei Icons angezeigt und der Test-Status in der Tabelle aktualisiert.

Ein grüner Haken zusammen mit dem Status ''APPLY'' (bzw. „FEASIBLE“) signalisiert, dass der Feasibility Check erfolgreich war.

Ein rotes Kreuz und der Status ''MANUALFEASIBILITY'' weisen darauf hin, dass entweder ein oder mehrere Kriterien nicht erfüllt wurden oder der Test in der Konfiguration noch nicht für die automatisierte Überprüfung freigeschaltet ist. In beiden Fällen muss der Test manuell von einem Mitarbeiter überprüft werden.

Tritt ein Fehler im Algorithmus auf, erscheint ein rotes Dreieck mit Ausrufezeichen. In diesem Fall bleibt der Test-Status unverändert oder wird auf ''FEASIBILITY\_REQUEST'' gesetzt, je nachdem wann und welcher Fehler auftritt (dieser Fall ist in der Abbildung \ref{fig:loading-with-results} nicht dargestellt).


\subsection{Modal-Page für detaillierte Ergebnisse des Feasibility Checks}

Nachdem der Feasibility Check für einige Tests abgeschlossen ist, können in einer zweiten Modal-Page detaillierte Ergebnisse eingesehen werden. Diese Seite wird geöffnet, wenn der Benutzer den grün umrandeten „Show Feasibility“-Button in der Startoberfläche (siehe Abbildung \ref{fig:whole-page}) betätigt. Abbildung \ref{fig:result-details} zeigt diese Modal-Page.

\begin{figure}[!htbp] 
    \centering 
    \makebox[\textwidth]{\includegraphics[width=0.85\paperwidth]{bilder/frontend/result-details.png}} 
    \caption{Modal-Page ''Submit Feasibility'' – Detaillierte Ergebnisse des Feasibility Checks} 
    \label{fig:result-details} 
\end{figure}

Das Popup-Fenster zeigt eine Tabelle mit allen Tests, ihren Informationen und den zugehörigen Ergebnissen. Der Benutzer kann einzelne Testzeilen aufklappen, um weitere Details zu erhalten. Links in der Tabelle sind Icons zu sehen, die anzeigen, welche Zeilen aufgeklappt werden können und welche aktuell geöffnet sind – dabei kann immer nur eine Zeile gleichzeitig ausgeklappt werden. 

In der aufgeklappten Ansicht werden in tabellarischen Form die gespeicherten \texttt{feasibility\_check\_item\_results}, des zugehörigen Tests angezeigt. Bei Tests die nur Überprüfungen mit dem Status-Ergebnis \texttt{''CHECK\_OFF''} aufweisen, liefern keine zusätzlichen Details, da in diesem Fall keine Feasibility-Resultate abgespeichert wurden. 

In der Spalte ''Check Item Type'' wird angegeben, ob es sich um einen Condition- oder Equipment Check handelt. In der Spalte ''State'' erscheint der Status, wie er in der Backend-Logik als Enum definiert ist. Zudem wird eine Beschreibung angezeigt, die dem Benutzer detaillierte Informationen zum jeweiligen Check liefert. Falls vorhanden, werden auch der zugehörige Operationstyp und die Parameter-ID dargestellt.

Diese Informationen sollen dem Benutzer helfen, die Ergebnisse nachvollziehen zu können. Außerdem ermöglichen sie es, bei unerwünschten Resultaten entsprechend zu reagieren. Entweder können die geplanten Tests und deren Parameter(-Werte) angepasst werden, oder weitere vom Feasibility Check abhängige Konditionen, wie etwa die Konfigurationstabelle können verändert werden.



