\chapter{Systemdesign}

Da der Feasibility Check als Funktionalität in die bestehende Software-Applikation \gls{REALIS} integriert werden soll, ist es notwendig, die bereits verwendeten Technologien und Programmiersprachen sowie die Architektur zu übernehmen, um eine nahtlose Integration zu gewährleisten.

Eine Evaluierung alternativer Technologien, die möglicherweise besser für die Applikation geeignet wären, wurde daher nicht durchgeführt. Stattdessen orientiert sich die Umsetzung strikt am bestehenden System.

Für das Datenbankdesign kommt eine Oracle-Datenbank mit SQL zum Einsatz. Die Backend-Logik wird in der objektorientierten Programmiersprache C\# implementiert. Für das Frontend wird das Webapplikationsframework Angular verwendet, das auf TypeScript basiert. Angular ermöglicht eine parallele Entwicklung sowohl für Webanwendungen als auch für native Mobilapplikationen.

\section{Architekturüberblick und Integration}

Die Abbildung \ref{fig:feasibility-check-komponentendiagramm} veranschaulicht die erweiterte REALIS-Architektur zur Realisierung des Feasibility Checks. Basierend auf dem bestehenden REALIS-Komponenten\-diagramm (siehe Abbildung 2.3) werden neue Komponenten hinzugefügt, um die Feasibility-Check-Funktionalität zu integrieren. Im Diagramm sind dabei aus Gründen der Übersichtlichkeit nur die wichtigsten Erweiterungen dargestellt.

\begin{figure}[!htbp]
    \centering
    \includegraphics[width=1\textwidth]{bilder/REALIS-Komponentendiagramm-mit-Erweiterungen2.png}
    \caption{Feasibility Check Architekturdesign-Erweiterung von \gls{REALIS}}
    \label{fig:feasibility-check-komponentendiagramm}
\end{figure}



In der \textbf{Datenbank} werden neue Tabellen hinzugefügt, die sowohl zur Konfiguration als auch zur Speicherung der Feasibility Check Ergebnisse dienen. Hierzu gehören die Tabellen \texttt{ctlg\_feasibility\_config}, \texttt{feasibility\_check\_item\_result} und \texttt{check\_item\_\-result\_\-list}.

Im \textbf{Backend} wird eine HTTP-GET-Methode (in Abb. \ref{fig:feasibility-check-komponentendiagramm} und \ref{fig:sequence-diagram} mit ''Helfer-Methode'' benannt) implementiert. Diese Methode ruft den Kern-Algorithmus, die Funktion \texttt{FeasibilityCheck()} auf, die die eigentliche Datenverarbeitung übernimmt, die Ergebnisse der Feasibility Check Überprüfung in der Datenbank abspeichert und diese über die ''Helfer-Methode'' an das Frontend liefert.

Im \textbf{Frontend} wird im \textit{REALIS-Web Subsystem} eine neue Modal-Page ''Submit for Feasibility Check'' integriert, die es Benutzern ermöglicht, den Feasibility Check zu initiieren. Nach der Überprüfung werden die Ergebnisse übersichtlich dargestellt und zusätzliche Details können bei Bedarf abgerufen werden. 

Durch diese Erweiterungen wird die Feasibility-Check-Funktion nahtlos in die bestehende REALIS-Architektur eingebettet und ermöglicht eine effiziente Interaktion mit den relevanten Benutzern und Systemkomponenten.

Die wichtigsten Interaktionen zwischen Benutzer (\gls{QM}), Frontend, Backend und der Datenbank sind im Sequenzdiagramm in Abbildung \ref{fig:sequence-diagram} dargestellt. Die einzelnen Komponenten werden in den folgenden Kapiteln detailliert beschrieben. 

\begin{figure}[!htbp]
    \centering
    \includegraphics[width=1\textwidth]{bilder/Feasibility-Sequenz-Diagramm.png}
    \caption{Sequenzdiagramm Feasibility Check}
    \label{fig:sequence-diagram}
\end{figure}




\section{Datenbankdesign}
Um die Erweiterung der Datenbank für den Feasibility Check zu erklären, wird zunächst auf die grundlegende Struktur der REALIS-Datenbank eingegangen. In Abbildung \ref{fig:realis-datenbankdesign} ist ein Ausschnitt der wichtigsten Tabellen und deren Beziehungen dargestellt, einschließlich der vorgenommenen Erweiterungen. Die nachfolgenden Erläuterungen nennen jeweils den zugehörigen Tabellennamen aus der Abbildung in Klammern.

\subsection{REALIS-Datenbank}

Wie bereits in Kapitel \ref{Subsec:project-lifecycle} beschrieben, besteht ein \textbf{REALIS-Projekt} (\texttt{relproject}), in dem neue Produkte qualifiziert werden, aus mehreren \textbf{Tests} (\texttt{tests}). Jeder Test besitzt einen eindeutigen \textbf{State} (\texttt{ctlg\_state\_type}), der zu Beginn auf ''NEW'' gesetzt wird und nach Abschluss den Status ''ARCHIVE'' erhält (vgl. Abbildung \ref{fig:realis-project-lifecycle}, rechte Spalte). Zudem wird jeder Test einer \textbf{Kategorie} bzw. einem \textbf{Test-Typ} (\texttt{ctlg\_test\-\_sub\_type}) zugeordnet.

Ein Test besteht aus mehreren aufeinanderfolgenden \textbf{Operationen} (\texttt{operation}), die nacheinander im \gls{RPT}-Labor abgearbeitet werden. Die Tabellen, die eng mit der Operation verknüpft sind, sind in Abbildung \ref{fig:realis-datenbankdesign} im grün umrandeten Bereich dargestellt. Jede Operation kann dabei einen oder mehrere \textbf{Parameter} (\texttt{op\_data}) enthalten, die beispielsweise die Umgebungsbedingungen definieren. Die konkreten Werte dieser Parameter sind in dem Eintrag \texttt{OPD\_PLAN} hinterlegt. Jeder Parameter wird zudem einer \textbf{Kategorie} (\texttt{ctlg\_op\_\-data\_type}) zugeordnet.  


\begin{figure}[!htbp]
    \centering
    \makebox[\textwidth]{\includegraphics[width=0.98\paperwidth]{bilder/REALIS-Datenbankmodell2.png}}
    \caption{REALIS Datenbankdesign}
    \label{fig:realis-datenbankdesign}
\end{figure}

Die Tabelle \texttt{op\_name\_name\_type} bestimmt welche Arten von Operationen es gibt, und legt indirekt über die Tabelle \texttt{ctlg\_op\_loc\_type} fest, welche Operations-Typen auf welchen Standort verfügbar sind. Darüber hinaus definiert die 
\texttt{op\_data\_param\_type} Tabelle, welcher Parameter bei welcher Operation auf welchem Standort für welchen Produkttypen (\texttt{ctlg\_product\_type}) verfügbar ist. Hierbei sind der Standort und der Produkttyp aber nur optionale Einträge, um generische, standortübergreifende und/oder produktübergreifende Einträge zu realisieren.

Jeder (Stress-)Parameter einer Operation, die einen konkreten Wert definiert (\texttt{OPD\_PLAN}) muss von einer Maschine \texttt{ctlg\_machine} umgesetzt werden (können). Die eng mit der Maschine verknüpften Tabellen, befinden sich in der Abbildung \ref{fig:realis-datenbankdesign} im blau umrandeten Bereich. Eine Maschine ist verknüpft mit einem Machinen-Test-Typen (\texttt{machine\_test\_sub\_type}), der die Machine einem Test-Typen zuweist (\texttt{ctlg\_test\_sub\_type}). Außerdem besitzen Maschinen verschiedene ''Factsheets'' (\texttt{machine\_factsheet}) mit einem Parameter (\texttt{ctlg\_factsheet\_parameter}) und einem Wert (\texttt{MFS\_VALUE}). Dieser Wert korrespondiert mit der PlanValue (\texttt{OPD\_PLAN}) des Operations-Parameters. Hierbei gibt es aber (meist) für jeden Operations-Parameter-Typen zwei zugehörige Maschinen-Factsheet-Parameter(-Typen), die angeben, was jeweils das Minimum und das Maximum der Maschine, bei diesem Parameter ist.


\subsection{Datenbankerweiterung mit Feasibility Check}

Für den Feasibility Check muss die Datenbank nun erweitert werden. Hierbei müssen die Parameter und die Logik des Feasibility Checks, die in Kapitel \ref{Subsec:ParameterdestechnischenFeasibilityChecks} beschrieben worden sind, ermöglicht werden. Zusätzlich dazu sollen die in Kapitel \ref{Chap:Anforderungen} besprochenen Anforderungen eingehalten werden.

Für den Feasibility Check soll der Wert des Parameters der (Stress-)Operation, zum einen auf Sinnhaftigkeit und zum anderen auf Durchführbarkeit überprüft werden. Dieser Wert entspricht im Datenbankdesign von \gls{REALIS} dem Eintrag \texttt{OPD\_PLAN} in der Tabelle \texttt{op\_data}, die für den Parameter steht.

Damit dieser Wert auf Sinnhaftigkeit überprüft werden kann, werden in der generischen Tabelle \texttt{opd\_data\_param\_type}, die Operationstyp, Parametertyp, Standort und Produkttyp miteinander verknüpft, zwei neue Einträge hinzugefügt. Diese zwei Einträge beinhalten Werte, die den ''sinnvollen'' Bereich festlegen. Der Eintrag \texttt{OPT\_CD\_MIN} legt dabei die untere Grenze des Bereichs fest, also den minimalen möglichen Wert. Und der Eintrag \texttt{OPT\_CD\_MAX} definiert die obere Grenze, also den maximalen Wert.

Ähnlich wird das ganze für die Überprüfung der Durchführbarkeit angelegt. Hierbei wird die generische Tabelle ebenfalls durch zwei neue Einträge für das Minimum (\texttt{OPT\_EQ\_MIN\_ID}) und das Maximum (\texttt{OPT\_EQ\_MAX\_ID}) erweitert. Diese beinhalten aber nicht direkt zwei Werte, sondern sind Referenzen auf zwei Maschinen-Parametertypen (\texttt{ctlg\_factsheet\_parameter}), wobei einer der beiden dem ''minimalen'' Parameter und der andere dem ''maximalen'' Parameter entpricht. Die tatsächlichen Werte stehen dann in der verknüpften \texttt{machine\_factsheet} Tabelle, in dem Feld \texttt{MFS\_VALUE}. Dies wird so gelöst, da, wie im vorherigen Kapitel schon beschrieben, für jeden Operations-Parameter-Typen jede Maschine, die diesen Parameter-Typen umsetzten kann, zwei korrespondierende FactSheet-Parameter(-Typen) besitzt, in die in der mit der Maschine verknüpften Machine-Factsheet Tabelle über einen Foreign-Key verwiesen wird. Diese zwei Machine-FactSheet(-Typen) beschrieben dabei den minimalen und maximalen ''Typen'', wobei der tatsächlichen minimale und maximale Wert, den die Maschine für diesen Typen umsetzten kann, in dem Machine-Factsheet steht.





\section{Backend-Logik}
Beschreibung der C\#-Implementierung, inklusive wichtiger Klassen und Methoden.
flussdiagramme
nhibernate, try catch blöcke, 

\begin{figure}[!h]
    \centering
    \makebox[\textwidth]{\includegraphics[width=0.85\paperwidth]{bilder/flowchart-feasibilitycheck-without-param-check.png}}
    \caption{Flowchart Feasibility Check - Condition Check }
    \label{fig:feasibility-check-condition-check}
\end{figure}

\section{Frontend-Design}
Überblick über die Angular-Anwendung, Struktur und Benutzeroberfläche.

asnychroner datenabruf