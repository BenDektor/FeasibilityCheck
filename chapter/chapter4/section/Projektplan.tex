\section{Projektplan}
Für die Entwicklung des automatisierten technischen Feasibility Checks wurde zu Beginn ein Projektplan in Form eines GANTT-Diagramms erstellt, der in Abbildung \ref{fig:roadmap} dargestellt ist. Dieser Plan ist in mehrere Bereiche (Zeilen) unterteilt und umfasst die Erarbeitung der Anforderungen, den Entwurf des Datenbankdesigns, die Backend-Entwicklung, die Frontend-Entwicklung sowie das Testen des Systems. Zusätzlich wurden für die Benutzer spezifische Meilensteine definiert, die in der unteren Zeile des Diagramms veranschaulicht werden.

\begin{figure}[!htbp]
    \centering
    \includegraphics[width=1\textwidth]{bilder/Roadmap.pdf}
    \caption{Feasibility Check Projektplan}
    \label{fig:roadmap}
\end{figure}

Die geplanten Maßnahmen für die ersten zwei Monate konnten weitgehend umgesetzt werden. In den darauffolgenden Monaten lag der Schwerpunkt auf der Überarbeitung bestehender Algorithmen im Backend, da wiederholt neue Ausnahmefälle identifiziert wurden. Auch das Frontend wurde mehrfach angepasst, um den aktuellen Anforderungen gerecht zu werden.

Aufgrund dieser Fokussierung auf die Optimierung und stabile Umsetzung des Condition Checks und des Equipment Checks konnten der Substrat Check sowie weitere geplante Checks nicht mehr realisiert werden, das heißt, dass alle optionalen Meilensteine (in Abb. 4.1 blau umrandet) nicht umgesetzt werden konnten.

Aktuell sind das Datenbankdesign und das Backend bereits vom Testing-System auf eine Vorbereitungsumgebung (''Staging'') ausgerollt worden und werden dort von einzelnen Anwendern getestet. Für das Backend wurden zudem Unittests entwickelt, was im Frontend aus Zeitgründen nicht mehr möglich war. Weitere geplante Meilensteine konnten letztlich aufgrund von Zeitbeschränkungen nicht umgesetzt werden.