\section{Architekturüberblick}

Die Abbildung 4.1 veranschaulicht die erweiterte REALIS-Architektur zur Realisierung des Feasibility Checks. Basierend auf dem bestehenden REALIS-Komponentendiagramm (siehe Abbildung 2.3) werden neue Komponenten hinzugefügt, um die Feasibility-Check-Funktionalität zu integrieren. Im Diagramm sind dabei aus Gründen der Übersichtlichkeit nur die wichtigsten Erweiterungen dargestellt.

\begin{figure}[!htbp]
    \centering
    \includegraphics[width=1\textwidth]{bilder/Feasibility-Komponentendiagramm.png}
    \caption{Feasibility Check Architekturdesign-Erweiterung von \gls{REALIS}}
    \label{fig:feasibility-check-komponentendiagramm}
\end{figure}

\todo{maybe add opdataparamtype}

In der \textbf{Datenbank} werden neue Tabellen hinzugefügt, die sowohl zur Konfiguration als auch zur Speicherung der Check-Ergebnisse dienen. Hierzu gehören die Tabellen \texttt{CTLG\_FEASIBILITY\_CONFIG}, \texttt{FEASIBILITY\_CHECK\_ITEM\_RESULT} und \texttt{CHECK\_ITEM\_\-RESULT\_\-LIST}.

Im \textbf{Backend} wird eine HTTP-GET-Methode mit der Bezeichnung \texttt{CheckFeasibility()} implementiert. Diese Methode ruft die Kernlogik der Funktion, \texttt{FeasibilityCheck()}, auf, die die eigentliche Datenverarbeitung übernimmt, die Ergebnisse in der Datenbank abspeichert und ein einfaches Ergebnis zurückliefert.

Im \textbf{Frontend} werden Erweiterungen in zwei Bereichen vorgenommen: Im \textit{REALIS-Web Subsystem} wird eine neue Modal-Page \glqq Submit for Feasibility Check\grqq{} integriert, die es Benutzern ermöglicht, den Feasibility Check zu initiieren. Zusätzlich wird in der Komponente \textit{REALIS-Web-Operator} eine neue Webpage \glqq Feasibility-Worklist\grqq{} hinzugefügt, welche die Ergebnisse des Feasibility Checks sowie offene Aufgaben übersichtlich darstellt.

Durch diese Erweiterungen wird die Feasibility-Check-Funktion nahtlos in die bestehende REALIS-Architektur eingebettet und ermöglicht eine effiziente Interaktion mit den relevanten Benutzern und Systemkomponenten.


