\section{Software-Entwicklungsprozess}
Software-Entwicklungsprozesse werden durch sogenannte Vorgehensmodelle beschrieben. Das Ziel solcher Modelle ist es, eine Hilfestellung bei der Organisation von Software-Entwicklungsprojekten zu geben und die Menge aller dabei anfallenden Aktivitäten in klare und verbindliche Arbeitsschritte aufzuteilen. 

Für dieses Projekt wird das Prototyping-Modell eingesetzt. In diesem Ansatz werden schrittweise Prototypen basierend auf den aktuellen Anforderungen entwickelt. Das daraufhin eingeholte Feedback von Auftraggebern oder Endanwendern ermöglicht eine kontinuierliche Verfeinerung der Anforderungen und eine schrittweise Verbesserung des Prototyps. Die Stadien dieses Modells werden in Abbildung~\ref{fig:Prototyping-Modell} nochmal verdeutlicht \cite{senarath2021waterfall}.

\begin{figure}[h!]
    \centering
    \includegraphics[]{bilder/Prototyping_Stages.jpg}
    \caption{Prototyping-Modell Stadien}
    \label{fig:Prototyping-Modell}
\end{figure}

Das Prototyping-Modell wurde eingesetzt, da die Anforderungen des technischen Feasibility Checks zu Beginn noch nicht eindeutig definiert waren und erst im Laufe der Entwicklung bestimmte Aspekte geklärt werden konnten. Zudem förderte dieser Ansatz den regelmäßigen Austausch zwischen mir als Entwickler und den Auftraggebern.