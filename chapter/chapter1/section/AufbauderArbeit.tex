\section{Aufbau der Arbeit}
Die Bachelorarbeit ist so strukturiert, dass nach diesem einleitenden Kapitel in Abschnitt \ref{Chap:TheoretischeGrundlagen} die wichtigsten theoretischen Grundlagen erklärt werden. 
Hierbei wird kurz das Unternehmen Infineon Technologies AG vorgestellt und anschließend die Halbleitertechnologie behandelt. Danach wird auf die Software-Applikation \gls{REALIS} eingegangen und der technische Feasibility Check erläutert.

Im nächsten Schritt werden im Kapitel \textit{\nameref{Chap:Anforderungen}} die technischen und funktionalen Anforderungen beschrieben, die durch das Prototyping-Modell schrittweise verbessert und bei Bedarf erweitert worden sind. Zusätzlich wird eine Stakeholderanalyse durchgeführt.

Zur Realisierung der Lösung wird ein Design erstellt, das dann im Kapitel \textit{\nameref{Chap:Implementierung}} umgesetzt wird. Das Kapitel Test beschreibt die Testmethodik. Abschließend werden die während der Arbeit aufgetretenen Herausforderungen reflektiert und ein Fazit gezogen. Darüber hinaus wird ein Ausblick auf potenzielle Optimierungen und Erweiterungen des Systems bzw. Algorithmus gegeben.