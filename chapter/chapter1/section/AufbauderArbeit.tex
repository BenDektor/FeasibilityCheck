\section{Aufbau der Arbeit}
Die Bachelorarbeit ist so strukturiert, dass nach diesem einleitenden Kapitel in Abschnitt \ref{Chap:TheoretischeGrundlagen} zuallererst, die wichtigsten theoretischen Grundlagen geklärt werden. 
Hierbei wird kurz Infineon und die Halbleitertechnologie erläutert, woraufhin auf die Software-Applikation \ac{REALIS} eingegangen wird und anschließend kurz erklärt wird, was der englische Titel "Feasibility Check" überhaupt meint. Auch wird noch auf die in der Bachelorarbeit verwendeten Programmiersprachen bzw. Software-Anwendungen und deren Nutzen eingegangen.

Daraufhin werden im Kapitel \nameref{Chap:Anforderungen} die technischen und funktionalen Anforderungen beschrieben, die durch das Prototyping-Modell schrittweise verbessert und bei Bedarf erweitert wurden. Zusätzlich wird eine kurze Stakeholderanalyse durchgeführt.

...