\section{Aufbau der Arbeit}
Die Bachelorarbeit ist so strukturiert, dass nach diesem einleitenden Kapitel in Abschnitt \ref{Chap:TheoretischeGrundlagen} die wichtigsten theoretischen Grundlagen erklärt werden. 
Hierbei wird kurz das Unternehmen Infineon Technologies AG vorgestellt und anschließend die Halbleitertechnologie behandelt. Danach wird auf die Software-Applikation \gls{REALIS} eingegangen und der technische Feasibility Check erläutert.

Im nächsten Schritt werden im Kapitel \textit{\nameref{Chap:Anforderungen}} die technischen und funktionalen Anforderungen beschrieben, die durch das Prototyping-Modell schrittweise verbessert und teilweise erweitert worden sind. Zusätzlich wird eine Stakeholderanalyse durchgeführt.

Zur Umsetzung der Lösung wird im Abschnitt \textit{\nameref{Chap:systemdesign-implementation}} ein detailliertes Design entwickelt und die entsprechende Implementierung beschrieben. Daraufhin folgt eine Darstellung der Testmethodik sowie die Evaluation des Systems. Abschließend werden die während der Arbeit gewonnenen Erkenntnisse und aufgetretenen Herausforderungen reflektiert, ein Fazit gezogen und ein Ausblick auf potenzielle Optimierungen sowie Erweiterungen des Systems bzw. Algorithmus gegeben.