\section{Zielsetzung der Arbeit}
In dieser Bachelorarbeit wird ein sogenannter technischer Feasibility Check, zu deutsch Machbarkeitsstudie, geplant und umgesetzt. Dieser ist Bestandteil einer größeren internen Software-Applikation namens \ac{REALIS}(diese wird genauer beschrieben in Kapitel~\ref{Sec:REALIS}), welches
im Zuge einer Migration \todo{was für eine Migration?} von einer Windows-Applikation zu einer Web-Applikation zusätzlich verbessert und modernisiert werden soll. Dabei ist der Feasibility Check ein neues Feature, dass zuvor manuell von Anwendern durchgeführt wurde, und nun automatisiert werden soll.

Der Feasibility Check ist eine Backend-Logik, geschrieben in der Programmiersprache C\#, die durch einen einfachen HTTP-Aufruf auf dem Frontend, einer Website, gestartet wird. Dabei greift der Algorithmus auf Daten in der Datenbank zu und liefert ein Ergebnis zurück.

Neben der Backend-Entwicklung werden in dieser Arbeit auch ein erweitertes Datenbankmodell konzipiert und zwei verschiedene Frontends mit Angular, für die Web- bzw. native App Darstellung, ausgearbeitet \todo{was noch alles?}