\section{Zielsetzung der Arbeit}
In dieser Bachelorarbeit wird ein sogenannter automatisierter technischer Feasibility Check, zu Deutsch ''Machbarkeitsstudie'', geplant und umgesetzt. Dieser ist Bestandteil einer größeren internen Software-Applikation namens \gls{REALIS}, welche
im Zuge einer Migration von einer Windows-Desktop-Applikation zu einer Web-Applikation zusätzlich verbessert und modernisiert werden soll. Dabei ist der Feasibility Check ein Feature, das zuvor manuell von Anwendern durchgeführt werden musste und nun automatisiert werden soll.

Der technische Feasibility Check soll nahtlos in das System \gls{REALIS} integriert werden. Dabei wird er als Backend-Logik in der Programmiersprache C\# umgesetzt. Diese Logik kann durch einen einfachen Aufruf vom Frontend aus gestartet werden, wobei der Algorithmus auf Daten in der Datenbank zugreift und dem Anwender anschließend ein Ergebnis zur Verfügung stellt.

Neben der Entwicklung eines Algorithmus im Backend, der mittels einer REST-API bereitgestellt wird, wird in dieser Arbeit auch ein erweitertes Datenbankmodell konzipiert, um bestimmte Einstellungen und Ergebnisse des Feasibility Checks zu speichern. Zusätzlich wird ein Frontend mit Angular ausgearbeitet, das über die REST-API den Algorithmus ausführt und das zurückgelieferte Ergebnis darstellt.