\chapter{Diskussion}
In diesem Kapitel werden die zentralen Aspekte der Systementwicklung reflektiert und diskutiert. Dabei werden die Herausforderungen, die während des Projekts aufgetreten sind, analysiert und ein Ausblick auf mögliche Weiterentwicklungen und Optimierungen des Systems gegeben, um dessen zukünftige Einsatzfähigkeit und Effizienz zu steigern. Die Diskussion dient dazu, die gemachten Erfahrungen zu strukturieren und Potenziale für zukünftige Arbeiten aufzuzeigen.
\section{Herausforderungen}

Eine erste Herausforderung und zentrale Aufgabe war die Ausarbeitung der Anforderungen (Requirements). Dies erforderte eine präzise Kommunikation mit den Stake\-holdern, um sicherzustellen, dass alle funktionalen und nicht-funktionalen Anforderungen eindeutig definiert und umfassend dokumentiert werden. 

Schwierig war zudem, sich in das umfangreiche und komplexe Datenbankmodell von \gls{REALIS} einzuarbeiten. Die Größe und Komplexität des Modells machten es notwendig, sich intensiv mit der Struktur und den Zusammenhängen der Daten auseinanderzusetzen, um die für den Feasibility Check relevanten Tabellen zu identifizieren. Nur durch dieses tiefgehende Verständnis konnte garantiert werden, dass die Erweiterungen effizient und korrekt in das System integriert werden konnten.

Zu den Herausforderungen zählte auch die Entwicklung eines stabilen Algorithmus, der in allen Szenarien zuverlässig funktioniert und kein unerwartetes Verhalten zeigt. Dabei war es wichtig, dass der Feasibility Check nicht nur korrekte Ergebnisse liefert, sondern auch sinnvolle Fehlermeldungen generiert, um potenzielle Probleme frühzeitig zu erkennen und zu behandeln. Dies erforderte eine sorgfältige Planung, Implementierung und umfangreiche Tests.

\section{Ausblick}

Das System bietet zahlreiche Möglichkeiten für Weiterentwicklungen und Verbesserungen. Eine potenzielle Erweiterung stellt die Implementierung zusätzlicher Checks, wie beispielsweise eines Substrat Checks oder eines Board Checks dar, um die Funktionalität des Systems zu vervollständigen. Zudem können die bestehenden Unittest-Fälle erweitert werden, um eine noch umfassendere Abdeckung der Codebasis zu gewährleisten und die Stabilität des Systems weiter zu steigern.

Ein weiterer wichtiger Schritt ist die ausführliche Durchführung von Tests mit tatsächlichen Endnutzern. Solche Tests liefern wertvolle Einblicke in die Benutzerfreundlichkeit und die praktische Anwendbarkeit des Systems. Zudem kann eine Zeitanalyse sinnvoll sein, um zu quantifizieren, wie viel Zeit durch die Automatisierung im Vergleich zur manuellen Bearbeitung eingespart werden kann. Dies würde den Nutzen des automatisierten Systems deutlich unterstreichen.

Auch die Erweiterung des Frontends bietet Potenzial. So könnte eine Benutzeroberfläche implementiert werden, die das Einfügen von Min- und Max-Werten bzw. Parametern für Equipment- und Condition Checks sowie die Konfiguration der Feasibility Checks ermöglicht. Zudem kann eine Funktion sinnvoll sein, die es den Nutzern erlaubt, nach Durchführung des Feasibility Checks aus den als geeignet bewerteten Maschinen eine endgültige Auswahl zu treffen. Dieser Auswahlprozess kann durch Machine-Learning-Algorithmen unterstützt werden, um die optimale Maschine basierend auf historischen Daten und spezifischen Kriterien vorzuschlagen. Solche Weiterentwicklungen steigern die Flexibilität und Benutzerfreundlichkeit des Systems sowie seine Einsatzmöglichkeiten signifikant.
