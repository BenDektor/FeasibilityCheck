\chapter{Diskussion}
In diesem Kapitel werden die zentralen Aspekte der Systementwicklung reflektiert und diskutiert. Dabei werden die Herausforderungen, die während des Projekts aufgetreten sind, analysiert und die daraus gewonnenen Lernerfahrungen zusammengefasst. Abschließend wird ein Ausblick auf mögliche Weiterentwicklungen und Optimierungen des Systems gegeben, um dessen zukünftige Einsatzfähigkeit und Effizienz zu steigern. Die Diskussion dient dazu, die gemachten Erfahrungen zu strukturieren und Potenziale für zukünftige Arbeiten aufzuzeigen.
\section{Herausforderungen}

Während der Entwicklung des Systems traten mehrere Herausforderungen auf, die bewältigt werden mussten. Eine der ersten und zentralen Aufgaben war die Ausarbeitung der Anforderungen (Requirements). Dies erforderte eine präzise Kommunikation mit den Stakeholdern, um sicherzustellen, dass alle funktionalen und nicht-funktionalen Anforderungen klar definiert und dokumentiert werden. Eine weitere große Herausforderung bestand darin, sich in das umfangreiche und komplexe Datenbankmodell von \gls{REALIS} einzuarbeiten. Die Größe und Komplexität des Modells machten es notwendig, sich intensiv mit der Struktur und den Zusammenhängen der Daten auseinanderzusetzen, um die für den Feasibility Check relevanten Tabellen zu identifizieren. Nur durch dieses tiefgehende Verständnis konnte sichergestellt werden, dass die Erweiterungen effizient und korrekt in das System integriert werden konnten.

Ein weiterer kritischer Punkt war die Entwicklung eines sicheren Algorithmus, der in allen Szenarien zuverlässig funktioniert und kein unerwartetes Verhalten zeigt. Hierbei war es wichtig, dass der Feasibility Check nicht nur korrekte Ergebnisse liefert, sondern auch sinnvolle Fehlermeldungen generiert, um potenzielle Probleme frühzeitig zu erkennen und zu behandeln. Dies erforderte eine sorgfältige Planung, Implementierung und umfangreiche Tests, um die Robustheit des Systems sicherzustellen.

\section{Lernerfahrungen}

Das Projekt bot zahlreiche Gelegenheiten, wertvolle Lernerfahrungen zu sammeln. Zunächst konnten meine Programmierkenntnisse in C\# und Angular sowie im Bereich des Datenbankdesigns deutlich gestärkt werden. Die praktische Anwendung dieser Technologien in einem realen Projektkontext vertiefte mein Verständnis und meine Fähigkeiten in der Softwareentwicklung.

Ein weiterer wichtiger Aspekt war die Erkenntnis, wie essenziell eine umfassende Dokumentation ist. Sie dient nicht nur als Referenz für das Entwicklungsteam, sondern erleichtert auch die Kommunikation mit Stakeholdern und die spätere Wartung des Systems. Zudem wurde deutlich, wie wichtig es ist, frühzeitig die Anforderungen klar zu definieren und zu dokumentieren. Gleichzeitig ist es hilfreich, bereits in frühen Phasen Prototypen in Form von Diagrammen oder ''Mock-ups'' zu erstellen, um das gemeinsame Verständnis im Team zu fördern und Missverständnisse zu vermeiden.

Ein weiterer zentraler Lernpunkt war die Bedeutung von sicherem Code. Durch die Implementierung von robusten Fehlerbehandlungsmechanismen und die Erstellung von qualitativ hochwertigen Unittests konnte die Zuverlässigkeit des Systems deutlich erhöht werden.  Darüber hinaus zeigte sich, wie wichtig eine gute Zusammenarbeit im Team ist. Die regelmäßige Abstimmung, das Teilen von Wissen und die gemeinsame Lösung von Problemen waren entscheidend für den Erfolg des Projekts.

\section{Ausblick}

Das System bietet zahlreiche Möglichkeiten zur Weiterentwicklung und Verbesserung. Eine mögliche Erweiterung wäre die Implementierung zusätzlicher Checks, wie beispielsweise eines Substrat Checks oder eines Board Checks, um die Funktionalität des Systems zu vervollständigen. Zudem sollten die bestehenden Unittest-Fälle erweitert werden, um eine noch umfassendere Abdeckung der Codebasis zu gewährleisten und die Stabilität des Systems weiter zu erhöhen.

Ein weiterer wichtiger Schritt wäre die ausführliche Durchführung von Tests mit tatsächlichen Endnutzern. Diese Tests könnten wertvolle Einblicke in die Benutzerfreundlichkeit und die praktische Anwendbarkeit des Systems liefern. Zudem wäre eine Zeitanalyse sinnvoll, um zu quantifizieren, wie viel Zeit durch die Automatisierung im Vergleich zur manuellen Bearbeitung eingespart werden kann. Dies würde den Nutzen des automatisierten Systems verdeutlichen. 

Ein weiterer Ausblick betrifft die Erweiterung des Frontends. Hier könnte eine Benutzeroberfläche implementiert werden, die das Einfügen von Min- und Max-Werten für Equipment- und Condition-Checks sowie die Konfiguration der Feasibility-Checks ermöglicht. Zudem wäre es sinnvoll, eine Funktion zu entwickeln, die es den Nutzern erlaubt, nach Durchführung des Feasibility Checks eine geeignete Maschine aus allen identifizierten und als passend bewerteten Maschinen auszuwählen. Diese ausgewählte Maschine könnte dann für den Stresstest verwendet werden. Solche Erweiterungen würden die Flexibilität und Benutzerfreundlichkeit des Systems weiter steigern und seine Einsatzmöglichkeiten signifikant erweitern.
