\chapter{Theoretische Grundlagen}\label{Chap:TheoretischeGrundlagen}
In diesem Abschnitt wird zunächst die Firma Infineon Technologies AG, in Kooperation mit der diese Arbeit entstanden ist, vorgestellt. Anschließend werden die theoretischen Grundlagen der Halbleitertechnologie geklärt - dem Hauptgeschäftsfeld der Infineon Technologies AG -, um den Nutzen der internen Software-Applikation \gls{REALIS} verständlich zu machen. Im darauffolgenden Kapitel, welches das System \gls{REALIS} näher bringen soll, wird auf dessen sogennanten Project Lifecycle, die Architektur und Statistiken eingegangen. Der technische Feasibility Check, der einen spezifischen Teil dieser Applikation darstellt, wird im letzten Teil des Abschnitts behandelt.

\section{Infineon Technologies AG}

Die Infineon Technologies AG zählt zu den weltweit führenden Herstellern von Halbleitern in den Bereichen Automotive, Power \& Sensor Systems, Green Industrial Power und Connected Secure Systems. Mit rund 58.600 Mitarbeitern ist das Unternehmen global tätig und betreibt insgesamt 84 Standorte \cite{infineon2024unternehmenspraesentation}. Einer dieser Standorte ist Regensburg mit mehr als 3000 Mitarbeitern, wo sowohl Entwicklung als auch Fertigung betrieben wird. Regensburg gilt dabei als Innovationslabor und Hightech-Fabrik, und ist der einzige Standort, an dem sowohl Frontend- als auch Backend-Produktion erfolgen. \cite{infineon2024regensburg}.

\section{Halbleitertechnologie}

Die Halbleitertechnologie ermöglicht es, elektronische Schaltungen vollständig in einem einzigen Herstellungsverfahren zu erzeugen. Dabei entstehen alle elektronischen Bauelemente und elektrischen Verbindungen auf einem monolithischen Halbleiterplättchen, das als integrierter Schaltkreis, auf Englisch \gls{IC} bezeichnet wird. Diese kleinen, dünnen Plättchen bestehen in der Regel aus Silizium und werden nach der Vereinzelung als Chips bezeichnet. Ein fertiger Wafer, aus dem diese Chips hergestellt werden, ist in Abbildung \ref{fig:Silizium-Wafer} zu sehen.

Halbleitermaterialien zeichnen sich durch ihre besondere Fähigkeit aus, elektrischen Strom nur unter bestimmten Bedingungen zu leiten. Anders als Metalle, deren Leitfähigkeit bei steigender Temperatur abnimmt, wird die Leitfähigkeit von Halbleitern mit zunehmender Temperatur exponentiell größer. Diese Eigenschaft, kombiniert mit der Möglichkeit, die Leitfähigkeit gezielt durch Dotierung – das Einbringen von Fremd\-atomen – zu steuern, macht Halbleiter so vielseitig einsetzbar.

Zur Herstellung integrierter Schaltkreise wird hochreines, monokristallines Halbleitermaterial benötigt, bei dem alle Atome in einer gleichmäßigen, durchgehenden Struktur angeordnet sind. Da solche Strukturen in der Natur nicht vorkommen, müssen sie technisch durch das ''Züchten'' von Kristallblöcken in Stangenform erzeugt werden. Diese Stangen werden in dünne Scheiben - sogenannte Wafer - geschnitten, die als Ausgangsmaterial für die Chip-Produktion dienen. Ein Wafer kann je nach Größe hunderte bis zehntausende Chips enthalten, die alle gleichzeitig hergestellt werden können \cite{lienig2023halbleitertechnologie}.

\begin{figure}[!h]
    \centering
    \includegraphics[width=0.9\textwidth]{bilder/SiC-Wafer-Infineon.jpg}
    \caption{Fertiger Silizium-Wafer mit Chips \cite{infineon2024dünnsterWafer}}
    \label{fig:Silizium-Wafer}
\end{figure}

Silizium ist das am häufigsten verwendete Material in der Halbleiterindustrie, weil es viele praktische Vorteile bietet. Es hat die perfekte Balance für den Einsatz in verschiedenen elektronischen Anwendungen und funktioniert gut bei normalen Betriebstemperaturen. Silizium bildet ein stabiles und zuverlässiges Isoliermaterial, das in Schaltkreisen vielseitig eingesetzt werden kann. Es leitet Wärme effizient ab, was wichtig ist, um eine Überhitzung zu vermeiden, besonders bei kleinen und leistungsstarken Chips. Außerdem lässt sich Silizium einfach in großen, reinen Kristallen herstellen, die für eine gleichmäßige Leistung in der Chipproduktion entscheidend sind.

Die Fertigung beginnt mit einem Rohwafer, auf dem im Front-End-of-Line (FEOL) alle Dotierungen erfolgen. Im darauf folgenden Back-End-of-Line (BEOL) werden abwechselnd isolierende und metallische Schichten aufgetragen und strukturiert, wodurch Leiterbahnen und Durchkontaktierungen entstehen. Die Strukturen moderner Chips sind dabei im Nano- bis Mikrometerbereich angesiedelt \cite{lienig2023halbleitertechnologie}.

Am Ende des Fertigungsprozesses werden die integrierten Schaltkreise (\gls{IC}), die in parallelen Reihen und Spalten auf der Oberfläche des Wafers angeordnet sind, wie in Abbildung \ref{fig:Silizium-Wafer} gut erkennbar, durch senkrecht verlaufende Schnitte voneinander getrennt. Durch diesen Schritt entstehen kleine, rechteckige, dünne Plättchen, die als Chips bekannt sind. Die Wafer, die aus den gezüchteten Kristallstäben mit Innenlochsägen ausgeschnitten werden, sind kreisrunde Scheiben und weisen typischerweise Durchmesser von 200 bis 450mm auf. Die Dicke moderner Wafer liegt heute bei knapp unter 1 mm. Fortschrittliche Technologien wie die der Infineon Technologies AG ermöglichen jedoch bereits die Fertigung und Verarbeitung ultradünner Silizium-Wafer mit einer Dicke von nur 20 Mikrometern \cite{infineon2024dünnsterWafer}.

Mit zunehmender Miniaturisierung der Halbleiterprozesse steigen die Herstellungskosten, da die Technologie komplexer wird. Jedoch sinkt durch die Verkleinerung der benötigte Platz pro Funktionseinheit auf dem Chip, was die höheren Prozesskosten kompensieren kann. So führt jede neue Chip-Generation dazu, dass mehr Leistung für den gleichen Preis erzielt wird, also eine höhere Funktionalität pro investiertem Geld \cite{lienig2023halbleitertechnologie}.
\section{REALIS}\label{Sec:REALIS}
Wird bei Infineon von einem Kunden ein neues Produkt angefordert oder entwickelt Infineon selbst ein neues Produkt, so muss dieses zunächst getestet und qualifiziert werden, bevor es in Masse produziert werden kann. Mit Produkt ist dabei ein fertiger Chip, der auf einem Wafer hergestellt wurde, gemeint. Für diese Zuverlässigkeits- bzw. Qualitäts-Tests wurde bei Infineon eine Software mit dem Namen \gls{REALIS} entwickelt. Dieses System umfasst die komplette Planung und Dokumentation der Durchführung und Ergebnisse dieser Tests. Das System beinhaltet eine gleichnamige Datenbank, in der alle wichtigen Informationen gespeichert werden.

\subsection{Projekt-Lebenszyklus}\label{Subsec:project-lifecycle}
Um ein neues Produkt zu testen, wird vom sogenannten \gls{QM} ein neues Projekt in \gls{REALIS} angelegt. Dieses befüllt er mit verschiedenen (Stress-)Tests, basierend auf vorhandenen Templates, die Arbeitsschritte (Operationen), Start- und Enddaten, Parameter der Operationen einzelner Tests und weitere Informationen enthalten. Dieser erste Schritt entspricht der obersten Zeile in Abbildung \ref{fig:realis-project-lifecycle} und bildet den Anfang eines REALIS Projekt-Lebenszyklus. 

Für jeden der folgenden Schritte wird in \gls{REALIS} der ``State``(Status) der Tests eines Projektes verändert und damit der Fortschritt dokumentiert. Dabei steht dieser zu Beginn immer auf  ``NEW`` und wird anschließend nach jedem der im Folgenden beschriebenen Schritte auf einen neuen ``State`` geändert (vgl. Abbildung \ref{fig:realis-project-lifecycle}, rechte Spalte). Welcher neue Zustand einem Test zugewiesen wird, wird dadurch entschieden, ob der beschriebene Schritt erfolgreich durchgeführt werden konnte oder nicht.

Im zweiten Schritt des Lebenszyklus weist der \gls{QM} das Projekt durch einen internen Mechanismus, einem ''State-Change'' des Projekts, einem sogenannten \gls{RPT}-Labor zu. 

In dem festgelegten \gls{RPT}-Labor validieren im Anschluss Mitarbeiter manuell die Richtigkeit der angelegten Tests und überprüfen daraufhin, ob sie die angelegten Stresstests des Projektes auch durchführen können. 
Für die Validierung der Tests werden die Stressparameter auf deren Sinnhaftigkeit überprüft. Die Frage der Durchführbarkeit hängt davon ab, ob im zugewiesenen \gls{RPT}-Labor Maschinen vorhanden sind, die in der Lage sind, die geforderten Stressoperationen auszuführen und die festgelegten Stressparameter einzuhalten. Zudem müssen diese Maschinen betriebsbereit sein – also weder in Benutzung noch außer Betrieb.

Diese Prüfungen bezeichnen den aktuellen technischen Feasibility Check, dessen Ergebnisse in \gls{REALIS} dokumentiert werden.
Falls für einige Operationen bzw. Tests keine gültigen Maschinen vorhanden sind, werden diese Tests an andere \gls{RPT}-Labore delegiert. Dadurch müssen die zu testenden Produkte jedoch von einem Labor zum anderen transportiert werden, was aufgrund der weltweiten Verteilung viel Zeit in Anspruch nehmen kann.

\begin{figure}[!h]
    \centering
    \includegraphics[width=1\textwidth]{bilder/realis-project-lifecycle.png}
    \caption{REALIS Project Lifecycle \cite{REALISWikiIntern}}
    \label{fig:realis-project-lifecycle}
\end{figure}

Anschließend wird ein ''Sample'', also eine kleine Stückzahl des Produktes, zum beauftragten \gls{RPT}-Labor geschickt. Das ''Sample'' wird in der Fachsprache auch als \textit{Test lot}, oder zu Deutsch: \textit{Test-Los} bzw. nur \gls{los} bezeichnet. Die Stückzahl wird dabei in \gls{REALIS} dokumentiert. Die  Daten der Testoperation werden dann final festgelegt und der Test-Status wird geändert.

Daraufhin erfolgt die planmäßige Durchführung der einzelnen Operationen der\linebreak Stresstests. Dabei werden der Fortschritt und die Ergebnisse von Labor-Mitarbeitern, sogenannten \glspl{operator}, in \gls{REALIS} dokumentiert. Treten während der Stresstests Probleme oder Fehler auf, werden diese an den \gls{QM} weitergeleitet, der über das weitere Vorgehen entscheidet.

Nachdem alle Tests vollständig durchgeführt und dokumentiert worden sind, muss der \gls{QM} die Ergebnisse prüfen und bewerten. Zum Schluss werden die Tests dann archiviert, wobei aus Gewährleistungsgründen die Chips und Testergebnisse 16 Jahre aufbewahrt werden müssen. Damit ist der Projekt-Lebenszyklus abgeschlossen.

\subsection{Architektur und Technologie}
Das ursprüngliche Frontend von REALIS war eine Windows-Desktop-Applikation, die sowohl vom \gls{RPT}-Labor-Mitarbeiter als auch vom \gls{QM} genutzt wurde. Im Zuge einer Modernisierung wird das System schrittweise zu einer Web-Applikation migriert. Zeitgleich erfolgt eine Aufteilung in zwei separate Anwendungen, eine für den \gls{QM} und eine für den \gls{operator}, mit dem Ziel, die Geschäftsprozesse zu vereinfachen und die Nutzerfreundlichkeit zu verbessern.

Abbildung \ref{fig:realis-komponentendiagramm} zeigt die aktuelle Systemarchitektur in Form eines Komponentendiagramms. Im Backend (grün dargestellt) kommuniziert der \texttt{REALIS-Server} über eine \texttt{DataAccess}-Schnittstelle direkt mit der zentralen \texttt{REALIS-Datenbank}, welche auf Oracle basiert. Der \texttt{REALIS-Server} stellt die Geschäftslogik (Business-Layer) bereit und wird über eine \texttt{REST-API} von den Frontends genutzt.

\begin{figure}[!h]
    \centering
    \includegraphics[width=1\textwidth]{bilder/REALIS-Komponentendiagramm.png}
    \caption{REALIS Komponentendiagramm}
    \label{fig:realis-komponentendiagramm}
\end{figure}

Das Frontend besteht aus zwei Hauptkomponenten (blau dargestellt):
\begin{enumerate}
    \item \textbf{REALIS-Desktop-Applikation:} \\
Die ursprüngliche Desktop-Anwendung, die sukzessive durch die Integration von neuen Web-Funktionalitäten modernisiert wird. Diese Web-Module, die mit dem Framework Angular entwickelt werden, sind als Subsystem (\texttt{REALIS-Web}) innerhalb der Desktop-Applikation eingebettet. Dieses System steht sowohl dem \gls{QM} als auch dem \gls{RPT}-Labor-Mitarbeiter zur Verfügung.

\item \textbf{REALIS-Web-Operator-System:} \\
Eine eigenständige Web-Applikation, die speziell für die Anforderungen des \gls{RPT}-Labors entwickelt wird. Diese Anwendung befindet sich noch in der Entwicklung, wird jedoch bereits für einige Aufgaben eingesetzt. Für nicht implementierte Funktionen muss der \gls{RPT}-Mitarbeiter vorübergehend auf die alte Desktop-Applikation ausweichen. Zusätzlich ist geplant, das \texttt{REALIS-Web-Operator-System} als native iOS-App für mobile Apple-Geräte (z. B. iPads, iPhones) bereitzustellen. Die Verwendung von Angular ermöglicht dabei eine plattformübergreifende Entwicklung, die sowohl als Web-App als auch als native App funktioniert.
\end{enumerate}

Das Diagramm \ref{fig:realis-komponentendiagramm} verdeutlicht die Trennung zwischen Backend und Frontend sowie die unterschiedlichen Nutzerrollen (\gls{QM} und \gls{RPT}-Employee), die spezifische Zugriffsrechte auf die jeweiligen Systeme haben.


\subsection{Weitere Funktionen und Statistiken}
Neben der Möglichkeit, Qualitätstests (Reliability-Tests) anzulegen, bietet \gls{REALIS} eine Vielzahl zusätzlicher Funktionen, die dazu beitragen, Prozesse effizienter zu gestalten und Engpässe zu vermeiden. So unterstützt das System beispielsweise die Planung individueller Laborkapazitäten, wodurch unnötige Investitionen vermieden und vorhandene Ressourcen optimal genutzt werden können. Darüber hinaus ermöglicht \gls{REALIS} die Referenzierung bereits durchgeführter Testergebnisse, um redundante Tests zu vermeiden und Zeit sowie Kosten zu sparen.

Nach Abschluss eines Tests können in \gls{REALIS} automatisch benötigte Ergebnisberichte generiert werden – sowohl für den Kunden als auch für das \gls{RPT}-Labor. Dies erleichtert die Dokumentation und erhöht die Effizienz im Testmanagement.

Seit 2001 ist \gls{REALIS} im Einsatz und verzeichnet derzeit etwa 4.300 aktive Nutzer. Das System findet Anwendung in 101 \gls{RPT}-Laboren in 17 Ländern. Aktuell verwaltet \gls{REALIS} rund 270.000 Projekte mit etwa 1,9 Millionen Stresstests \cite{REALISPowerPointIntern}.
\section{Technischer Feasibility Check}
Der automatisierte technische Feasibility Check, bezeichnet eine ''Machbarkeitsprüfung'' oder ''Machbarkeitsstudie'', die bewertet, ob ein Stresstest an einem \gls{RPT}-Laborstandort durchgeführt werden kann.

Der Check ist ein integraler Bestandteil des \gls{REALIS}-Projekt-Lifecycles, wie in Kapitel \ref{Subsec:project-lifecycle} beschrieben. Derzeit wird dieser manuell von Mitarbeitern des \gls{RPT}-Labors durchgeführt und in \gls{REALIS} dokumentiert. Dafür prüfen sie, ob die im Labor vorhandenen Kapazitäten und Bedingungen ausreichen, um die geforderten Tests für das jeweilige Produkt durchzuführen.

Die Beurteilung basiert auf mehreren Parametern. Um diese besser zu verstehen, wird im folgenden Kapitel erläutert, wie ein Stresstest im Detail abläuft und welche Schritte dabei erforderlich sind.


\subsection{Stresstests (Reliability Tests)}

Stresstests, auch als \textit{Reliability Tests} bezeichnet, sind ein wesentlicher Bestandteil der Qualifikation neuer Produkte. Sie dienen dazu, die Belastbarkeit und Zuverlässigkeit eines Produkts unter definierten Bedingungen zu überprüfen. Zu den häufigsten Testarten zählen Temperaturtests (Hitze- und Kältetests), Drucktests, Feuchtigkeitstests und elektrische Tests – häufig auch in Kombination.

Jeder Stresstest setzt sich aus mehreren aufeinanderfolgenden Operationen zusammen, die von einem \gls{RPT}-Labor-\gls{operator} mithilfe eines Test-\glspl{lot} (einer kleinen Stückzahl des Produkts) durchgeführt werden. Üblicherweise beginnt ein Test mit einer ''START''-Operation und endet mit einer ''END''-Operation. Dazwischen erfolgen mehrere Stressoperationen, die den spezifischen Stresstyp des Tests abbilden und die eigentliche Belastungsprüfung darstellen. Nach jeder Stressoperation wird das \gls{lot} einer Funktionsprüfung (weitere Operation) unterzogen, um sicherzustellen, dass alle Chips die vorhergehende Belastung unbeschadet überstanden haben.

Während der Durchführung der Stressoperationen platziert der \gls{operator} das \gls{lot} in eine geeignete Testmaschine und passt die Parameter gemäß den geplanten Vorgaben in \gls{REALIS} an. Dabei kann eine Operation mehrere verschiedene Parameter enthalten, mit vordefinierten Werten. Beispielsweise wird bei Temperaturtests die Zieltemperatur sowie die Testdauer eingestellt, während bei anderen Tests, wie etwa mechanischen Belastungen, die Anzahl der Wiederholungen definiert wird. Die Chips des \glspl{lot} werden dabei häufig auf einem sogenannten \gls{board} montiert, das als Träger fungiert und Leiterbahnen enthält, um die elektrische Verbindung zwischen dem Testsystem und den Chips herzustellen.

Für kleinere Chips kommt oft ein zusätzliches Zwischenelement, das sogenannte \gls{substrate}, zum Einsatz. Das \gls{substrate} fungiert als Trägerelement und enthält ebenfalls Leiterbahnen. Es dient dazu, die Kontakte des Chips zu „vergrößern“ und somit die Handhabung und Verbindung mit dem \gls{board} zu erleichtern.

In einigen Testverfahren, wie beispielsweise in Regensburg, wird jedoch kein \gls{board} benötigt. Hier werden die Chips in eine Karte (vergleichbar mit einer EC-Karte) integriert, die selbst als \gls{substrate} dient. Die Karte wird in einer Testmaschine wiederholt mechanischen Belastungen, wie Biegungen, ausgesetzt. Die Karten werden dabei einzeln in die Maschine eingelegt, was den Einsatz eines \glspl{board} überflüssig macht.

Alle durchgeführten Operationen eines Tests müssen vom \gls{operator} in \gls{REALIS} dokumentiert werden, insbesondere die Stressoperationen und die anschließenden Funktionsprüfungen. Dabei werden unter anderem die Anzahl der Chips, die nach einer Stressoperation ausgefallen sind, sowie die vermuteten Ursachen des Ausfalls festgehalten.
Nur wenn das gesamte \gls{lot} unbeschadet ist kann mit der nächsten Operation fortgefahren werden, ansonsten wir der Test vorerst gestoppt und der \gls{QM} muss entscheiden, wie weiter vorgegangen wird.

\subsection{Parameter des technischen Feasibility Checks}\label{Subsec:ParameterdestechnischenFeasibilityChecks}

Im Rahmen des technischen Feasibility Checks wird zunächst geprüft, ob die vom \gls{QM} definierten Parameter und Werte der geplanten Stressoperationen sinnvoll und durchführbar sind. Dies umfasst beispielsweise die Überprüfung, ob die vorgegebene Testtemperatur im technisch möglichen Bereich liegt. Zudem wird kontrolliert, ob für jeden Parameter ein geeigneter Wert festgelegt wurde, falls dies erforderlich ist.

Anschließend wird geprüft, ob am zuständigen \gls{RPT}-Labor geeignete Stressmaschinen für den geforderten Stresstesttyp verfügbar sind. Dabei wird sichergestellt, dass die Maschinen einsatzbereit, nicht defekt und auch nicht in Wartung sind. Außerdem wird überprüft, ob sie die vorgegebenen Parameter und Werte tatsächlich umsetzen können. Zusätzlich wird kontrolliert, ob passende \glspl{board} und \glspl{substrate} für die Chips des \glspl{lot} vorhanden sind, sofern diese benötigt werden.

Aktuell muss ein Mitarbeiter des Labors alle diese Prüfungen für jeden Stresstest manuell durchführen. Bei wiederkehrenden Tests kann es sein, dass der Mitarbeiter bereits aus Erfahrung weiß, dass der Test durchführbar ist, und der Feasibility Check somit direkt im System bestätigt werden kann. In anderen Fällen kann es jedoch mehrere Stunden dauern, um die Machbarkeit eines Tests festzustellen.

Um diesen Prozess zu beschleunigen und eine höhere Effizienz zu erreichen, soll dieser Schritt des \nameref{Subsec:project-lifecycle} von \gls{REALIS} automatisiert werden. Durch diese Automatisierung wird nicht nur die Dauer des Feasibility Checks reduziert, sondern auch eine fundiertere Begründung für die Entscheidung ermöglicht, da das System aktuell nur eine einfache „Ja/Nein“-Auswahl zulässt. Darüber hinaus trägt die Automatisierung dazu bei, menschliche Fehler zu minimieren und die Genauigkeit der Entscheidungen zu erhöhen.